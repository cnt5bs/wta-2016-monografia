\section{Geração de Autômatos UBA}
\label{sec:isv:geracao}

Embora respeitando o modelo formal de um autômato adaptativo, o projeto do analisador semântico de especificações LUBA envolve uma etapa de natureza criativa, artística. Silva reforça esta posição, enfatizando a dificuldade de se sistematizar o projeto desta etapa dos programas de tradução:

   \begin{quote}
   ``Devido a não existir um padrão para representar a semântica estática de uma linguagem, bem como pela quantidade e tipos de análise variarem de uma linguagem para outra, implementar a análise semântica torna-se uma tarefa não muito bem definida''~\cite{silva:2011}.
   \end{quote}

Com a ajuda da ferramenta ANTLR4, torna-se possível mecanizar a escrita do código que implementa a análise sintática de especificações LUBA. O diagrama UML apresentado na Fig~\ref{fig-antlr-2-dat} revela como este código se integra na estrutura do programa de geração de UBAs. A linha com um losango conectando a classe \uml{Analisador Sintático LUBA} com a classe \uml{Analisador Semântico LUBA} representa a ideia de constituição: o analisador sintático é constituido, estruturalmente, por um analisador semântico. Da mesma forma interpreta-se a conexão entre a classe \uml{Gerador UBA} e o analisador sintático. Ou seja, a classe dos geradores de UBA absorve o analisador sintático na sua estrutura, originando o programa que traduz especificações de autômatos LUBA em máquinas virtuais UBA.

      \figTop[0.6]{\itaPastaFig}{fig-antlr-2-dat}{Diagrama UML revelando a estrutura de classes do \uml{Gerador UBA}.}

Quando em operação, o gerador de UBAs traduz especificações LUBA em máquinas virtuais UBA, programadas na linguagem Groovy (Fig~\ref{fig-antlr-3-dat}). É o modelo de geração de máquinas UBA que será detalhado na próxima seção.

      \figLatHere[0.6]{\itaPastaFig}{fig-antlr-3-dat}{Diagrama de atividades UML da geração de uma máquina virtual UBA a partir da especificação LUBA de um autômato. O processo de tradução é realizado pelo \uml{Gerador UBA}.}

\subsection{Análise Semântica de Especificações LUBA}

Retomando a primeira regra de produção da gramática LUBA (apresentada na Seção~\ref{sec:isv:gramatica}), tem-se:

\begin{lstlisting}[style=antlr]
luba : UBA idClasse ABBL declaracao* FEBL ;
\end{lstlisting}

De forma mecânica, a ferramenta ANTLR4 produz o código do analisador sintático. A cada não-terminal da gramática processada, ele gera um par de métodos de entrada e de saída (do não-terminal). O corpo destes métodos é que devem implementar o modelo semântico da linguagem LUBA. Assim, par-e-par com o não-terminal \naoTerminal{luba}, existem os métodos \codigo{enterLuba()} e \codigo{exitLuba()}. Por uma decisão de projeto, apenas o primeiro método será utilizado para instanciar um novo objeto (semântico) da classe \codigo{Uba}:

\begin{lstlisting}
<<GeradorUba::ação semântica 'luba'>>=
@Override
public void enterLuba(UbaParser.LubaContext ctx) {
    novaUba = new gerador.Uba()
}
def novaUba // variável que referencia a "nova" uba
\end{lstlisting}

   \begin{quote}
   \textit{Observação para a Comissão do WTA: a parte do texto a seguir está INCOMPLETA e contém apenas as ideias e código centrais. O texto precisa ser refinado! Isto se aplica para as seções \ref{sec:isv:geracao-regra}, \ref{sec:isv:geracao-funcao}, \ref{sec:isv:geracao-consulta}, \ref{sec:isv:geracao-remocao} e \ref{sec:isv:geracao-estados}.}
   \end{quote}

\subsection{Geração de Regra Adaptativa}
\label{sec:isv:geracao-regra}

\begin{lstlisting}[style=antlr]
decRegra :
	cfa?
	ABPAR decEstOrg VIRGULA idSimbolo? FEPAR SETA decEstDst
	cfp?
	;
\end{lstlisting}

\begin{lstlisting}
<<GeradorUba::ação semântica 'decRegra'>>=
@Override
public void exitDecRegra(UbaParser.DecRegraContext ctx) {
    def cfa = ""
    if (ctx.cfa()) {
        cfa = ctx.cfa().idFuncao().getText()
    }
    def origem = ctx.decEstOrg().idEstado().getText()
    def estimulo = ""
    if (ctx.idSimbolo()) {
        estimulo = ctx.idSimbolo().getText()
    }
    def destino = ctx.decEstDst().idEstado().getText()
    def cfp = ""
    if (ctx.cfp()) {
        cfp = ctx.cfp().idFuncao().getText()
    }
    novaUba.incorporarRegraAdaptativa(cfa, origem, estimulo, destino, cfp)
}
\end{lstlisting}

\subsection{Geração de Função Adaptativa}
\label{sec:isv:geracao-funcao}

Retomando o exemplo apresentado na Seção~\ref{sec:isv:regra-adaptativa} ...

\begin{lstlisting}
uba Exemplo2 {
   fx => (>q1, a) -> !q1
   (q1, ) -> q2
   (q2, ) -> !q3
	<!-- restante da especificação -->
}
\end{lstlisting}

Em seguida, a instrução contém uma chamada da função adaptativa \lstinline|fx|. Chamadas de tais funções serão refinadas na Seção~\ref{sec:isv:chamada-funcao-adaptativa}. Além de vincular os identificadores de estados, a seção de regras da classe de dispositivos adaptativos também introduz as seguintes informações na tabela de transições de \lstinline|a2|:

\begin{center}\begin{tabular}{c c c c c c c}
Identificador   & Tipo  & CFA    & Corrente & Entrada & Seguinte & CFP \\
\hline
\lstinline|t1|	& Transição & fx  & \lstinline|q1| & \lstinline|a| & \lstinline|q1| & \\
\lstinline|t2|	& Transição &   & \lstinline|q1| & & \lstinline|q2| & \\
\lstinline|t3|	& Transição &   & \lstinline|q2| & & \lstinline|q3| & \\
\end{tabular}\end{center}

\begin{lstlisting}[style=antlr]
decFuncao : idFuncao
	ABBL
	seqConsultas?
	seqRemocoes?
	seqInsercoes?
	FEBL
	;
\end{lstlisting}

\begin{lstlisting}
<<GeradorUba::ação semântica 'decFuncao'>>=
@Override
public void enterDecFuncao(UbaParser.DecFuncaoContext ctx) {
    funcaoAdaptativa = new gerador.FuncaoAdaptativa()
    def funcaoId = ctx.idFuncao().getText()
    funcaoAdaptativa.cabecalho( funcaoId )
}
def funcaoAdaptativa

@Override
public void exitDecFuncao(UbaParser.DecFuncaoContext ctx) {
    funcaoAdaptativa.encerrar()
    novaUba.incorporarFuncao(funcaoAdaptativa)
}
\end{lstlisting}

\subsection{Geração de Ação de Consulta}
\label{sec:isv:geracao-consulta}

Retomando o exemplo apresentado na Seção~\ref{sec:isv:chamada-funcao-adaptativa} ...

Desta maneira, monta-se a tabela de símbolos para a execução da função adaptativa \lstinline|fx|:

  \begin{center}\begin{tabular}{c c c c c }
  Identificador   & Tipo  & Escopo    & Referência \\
  \hline
  \lstinline|@x|	& Estado & LOCAL  & \lstinline|q1| \\
  \lstinline|@y|	& Estado & LOCAL  & \lstinline|q3| \\
  \end{tabular}\end{center}

\begin{lstlisting}[style=antlr]
acaoConsulta : BARRA padraoRegra BARRA ;
\end{lstlisting}

\begin{lstlisting}
<<GeradorUba::ação semântica 'acaoConsulta'>>=
@Override
public void enterAcaoConsulta(UbaParser.AcaoConsultaContext ctx) {
    acao = new gerador.AcaoConsulta()
}
def acao

@Override
public void exitAcaoConsulta(UbaParser.AcaoConsultaContext ctx) {
    def cfa = ""
    if (ctx.cfa()) {
       cfa = ctx.cfa().idFuncao().getText()
    }
    def cfp = ""
    if (ctx.cfp()) {
        cfp = ctx.cfp().idFuncao().getText()
    }
    acao.fechar( funcaoAdaptativa, cfa, cfp )
}
\end{lstlisting}

\subsection{Geração de Ações de Remoção}
\label{sec:isv:geracao-remocao}

Retomando o exemplo apresentado na Seção~\ref{sec:isv:chamada-funcao-adaptativa} ...

Embora não ocorra alteração na tabela de estados da instância \instancia{a2}, a situação final da sua tabela de transições se altera para:

   \begin{center}\begin{tabular}{c c c c c c c}
   Identificador   & Tipo  & CFA    & Corrente & Entrada & Seguinte & CFP \\
   \hline
   \lstinline|t1|	& Transição & fx  & \lstinline|q1| & \lstinline|a| & \lstinline|q1| &
   \end{tabular}\end{center}

\subsection{Geração de Estados}
\label{sec:isv:geracao-estados}

Retomando o exemplo apresentado na Seção~\ref{sec:isv:subjacente} ...

\begin{lstlisting}
uba Exemplo1 {
	( >q1, a ) -> q1
	( q1, b ) -> q2
	( q2, c ) -> !q3
}
\end{lstlisting}

Todas estas observações podem ser representadas em uma tabela de símbolos:

\begin{center}\begin{tabular}{c c c c c}
Identificador   & Tipo  & Escopo    & Inicial & Final \\
\hline
\lstinline|q1|	& Estado & INSTÂNCIA  & S & N \\
\lstinline|q2|	& Estado & INSTÂNCIA  & N & N \\
\lstinline|q3|	& Estado & INSTÂNCIA  & N & S \\
\end{tabular}\end{center}

\begin{quote}\textit{(Por exemplo, considere a seguinte regra da gramática, referente à declaração de um novo estado de dispositivo:)}\end{quote}

\begin{lstlisting}[style=antlr]
estadoInstancia : ( INICIAL | FINAL )? idEstado ;
\end{lstlisting}

\begin{lstlisting}
<<GeradorUba::ação semântica 'estadoInstancia'>>=
@Override
public void exitEstadoInstancia(UbaParser.EstadoInstanciaContext ctx) {
    def estadoId = ctx.idEstado().getText();
    def primeiro = false
    if (ctx.INICIAL()) primeiro = true
    def fim = false
    if (ctx.FINAL()) fim = true
    acao.condicao( estadoId, primeiro, fim )
}
\end{lstlisting}

