\section{Regras Adaptativas}
\label{sec:isv:regra-adaptativa}

%\begin{quote}\textit{(A ideia central desta seção é a de apresentar um modelo semântico de dispositivo adaptativo usando uma notação personalizada descrita na seção anterior. Mais adiante, tal notação será utilizada para especificar a semântica das Unidades Biomiméticas Adaptativas (UBA), estudadas no Cap.~\ref{cap:isv:uba}. Como base, considera-se o artigo \textit{An adaptive automata operational semantic} \cite{italo:2009} e \textit{Especificações adaptativas e objetos, uma técnica de design de software a partir de statecharts com métodos adaptativos} \cite{vega:2012}.)}\end{quote}

Como modelo de computação, o autômato adaptativo é equivalente à Máquina de Turing~\cite{neto:1998}. Por conseguinte, o seu modelo semântico pode ser especificado com elementos das linguagens de programação de alto nível. Em particular, dando continuidade aos elementos notacionais anteriormente apresentados, serão incorporados o suporte à chamada de funções adaptativas nas regras de transição, bem como o suporte à descrição dos seus efeitos por meio de ações adaptativas. O diagrama mostrado na Fig.~\ref{fig-regra-adaptativa-uxf} enfatiza a principal relação entre as ações adaptativas e o espaço de regras adaptativas do autômato: modificação. Uma ação adaptativa pode modificar as regras de um autômato durante a sua opearação. É esta capacidade de auto-modificação espontânea de comportamento que caracteriza a adaptatividade.

	\figLatHere[0.67]{\itaPastaFig}{fig-regra-adaptativa-uxf}{Modelo conceitual de um autômato adaptativo com destaque para as suas ações adaptativas.}

Embora ofereça um suporte reduzido à adaptatividade, a linguagem LUBA suporta a realização de experimentos que ajudam a apresentar os conceitos utilizados na concepção de modelos biomiméticos. Considere-se a especificação de um novo tipo de autômato contendo uma única regra adaptativa inicialmente:

\begin{lstlisting}
<<exemplo2.uba>>=
uba Exemplo2 {
    fx => (>q1, a) -> q1
    (q1, ) -> q2
    (q2, ) -> !q3
    <<Exemplo2::especificação da função adaptativa>>
}
\end{lstlisting}

\noindent
A instanciação de um autômato desta classe é feita nos mesmos moldes da classe \lstinline|Exemplo1| anteriormente discutida:

\begin{lstlisting}
<<apl2exemplo::instanciação de uma Uma>>=
def d2 = new cenario.Exemplo2()
\end{lstlisting}

\noindent
A Fig~\ref{fig-cenario-Exemplo2-0-delta} contém uma representação visual da topologia deste autômato após ser instanciado. O seu estado inicial é \estado{q1} (ou seja, é o estado corrente imediatamente após a instanciação do autômato). A partir deste estado, duas regras podem ser disparadas para levar o autômato para os estados \estado{q1} e \estado{q2} (este último, um estado interno). Uma terceira regra contém uma transição que, ao ser disparada, leva \estado{a2} do estado \estado{q2} para o estado \estado{q3}. Tanto esta quanto a segunda regra contém o que se chama de \textit{transição em vazio} ou \textit{transição}-$\epsilon$. A primeira regra, entretanto, traz uma novidade: uma chamada da função adaptativa \funcao{fx}. Por este motivo, esta é uma regra adaptativa.

	\figLatHere[0.35]{\itaPastaFig}{fig-cenario-Exemplo2-0-delta}{Configuração inicial do dispositivo \uba{a2} na qual a auto-transição provoca uma chamada da função adaptativa \funcao{fx}, caso o símbolo de entrada seja uma ocorrência de \simbolo{a}. O estado \estado{q1} é inicial e corrente, enquanto o estado \estado{q2} é interno e \estado{q3}, final.}

Como se comporta o autômato \uba{a2} durante o processamento de uma cadeia de símbolos? Certamente ele reconhece a cadeia vazia denotada por $\epsilon$. Ele reconhece a cadeia $a$? A resposta irá depender do efeito provocado pela chamada da função adaptativa \funcao{fx}. A ideia é especificar o autômato adaptativo para que ele seja capaz de reconhecer a linguagem contendo todas as cadeias com a forma da expressão regular $\mathbf{a^nb^{2n}c^{3n}}$, para $n \geq 0$ (exemplo inspirado em uma apresentação feita por Neto~\cite{neto:2008}). Como anteriormente feito para o \lstinline|Exemplo1|, denota-se a classe de dispositivos adaptativos \lstinline|Exemplo2| por $\textit{Exemplo2}=(Q, \Sigma, q_0, F, \delta, \textit{CA})$ onde:

\begin{itemize}
\item $Q$ é um conjunto finito de estados;
\item $\Sigma$ é um conjunto finito de estímulos de entrada;
\item $q_0 \in Q$ é o estado inicial;
\item $F\subseteq Q$ é o conjunto de estados finais;
\item $\delta:Q\times\Sigma \rightarrow Q$ é a função de transição de estados;
\item \textit{CA} é a camada adaptativa de \textit{Exemplo2};
\end{itemize}

\noindent e

	\begin{align*}
	Q = \{q_1, q_2, q_3\},\ \Sigma = \{a\},\ q_0 = q_1,\ F = \{q_3\} \\
	\\
	\delta \triangleq \{ \mathsf{fx}\cdot(q_1, a) \mapsto q_1,  (q_1, \epsilon) \mapsto q_2,  (q_2, \epsilon) \mapsto q_3 \} \tag{i}\label{d2-transicoes}
	\end{align*}

\noindent
Como no exemplo \uba{Exemplo1}, a função de transição  $\delta$ contém três pares, mas o primeiro deles introduz uma chamada da função adaptativa \funcao{fx}. Agora, qual deve ser a definição desta função de modo que \uba{a2} reconheça as cadeias denotadas pelas expressão $\mathbf{a^nb^{2n}c^{3n}}$? Um novo mecanismo se faz presente na linguagem LUBA: definição de funções adaptativas.


\begin{lstlisting}
<<Exemplo2::especificação da função adaptativa>>=
fx {
    consultas
    / ( @x, )-> q2 /
    / ( q2, )-> @y /

    remoções
    / ( @x, ) -> q2 /
    / ( q2, ) -> @y /

    inserções
    / ( @x, b ) -> ^0 /
    / ( ^0, b ) -> ^1 /
    / ( ^1,   ) -> q2 /
    / ( q2,   ) -> ^2 /
    / ( ^2, c ) -> ^3 /
    / ( ^3, c ) -> ^4 /
    / ( ^4, c ) -> @y /
}
\end{lstlisting}
