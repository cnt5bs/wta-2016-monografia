# FuncaoAdaptativa {-}

\begin{lstlisting}
<<FuncaoAdaptativa.groovy>>=
package gerador

class FuncaoAdaptativa {

    FuncaoAdaptativa() {
        locais = []
        acoesConsulta = []
        acoesRemocao = []
        acoesInsercao = []
        maiorG = 0
    }
    def id
    def locais
    def acoesConsulta
    def acoesRemocao
    def acoesInsercao
    def maiorG
    def corpo

    def cabecalho( id ) {
        this.id = id
    }

    def encerrar() {
        corpo = []
        corpo << "def ${id}() {"
        locais.each { corpo << "def $it" }
        corpo << "// Consultas..."
        acoesConsulta.each { corpo += it.corpo }
        corpo << "// Remoções..."
        corpo << "def retirar = [:]"
        acoesRemocao.each { corpo += it.corpo }
        corpo << "def restantes = regras - retirar"
        corpo << "// Inserções..."
        corpo << "regras = [:]"
        corpo << "def t = 1"
        corpo << "restantes.each { i, r -> regras[t] = r; t++ }"
        corpo << "//---"
        acoesInsercao.each { corpo += it.corpo }
        corpo << "g = g + ${maiorG+1}"
        //        locais.each { v -> corpo << "println $v" }
        corpo << "}\n\n"
    }



    def atualizarG( estado ) {
        if( maiorG < estado ) maiorG = estado
    }

    def adicionarLocal( estado ) {
        if( ! (estado in locais) ) {
            locais << estado
        }
    }

    def adicionarConsulta( acao ) {
        acoesConsulta << acao
    }

    def adicionarRemocao( acao ) {
        acoesRemocao << acao
    }

    def adicionarInsercao( acao ) {
        acoesInsercao << acao
    }
}
\end{lstlisting}


