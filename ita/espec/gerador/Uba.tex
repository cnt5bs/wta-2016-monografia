# Uba {-}

\begin{lstlisting}
<<Uba.groovy>>=
package gerador

class Uba {

    Uba() {
        tabRegrasAdaptativas = new TabRegrasAdaptativas()
        tabEstados = new TabEstados()
        tabFuncoes = new TabFuncoesAdaptativas()
    }
    def nome
    def tabEstados
    def tabRegrasAdaptativas
    def tabFuncoes

    def incorporarEstado( id ) {
        return tabEstados.incorporar( id, Tipo.ESTADO, Escopo.INSTANCIA, false, false )
    }

    def inicial( id ) {
        tabEstados.inicial( id )
    }

    def terminal( id ) {
        tabEstados.terminal( id )
    }

    def estado( id ) {
        tabEstados.estado( id )
    }

    def incorporarRegraAdaptativa(cfa, origem, estimulo, destino, cfp) {
        tabRegrasAdaptativas.incorporar(cfa, origem, estimulo, destino, cfp)
    }

    def incorporarAcao( id,  instrucao ) {
        tabFuncoes.incorporarAcao(id, instrucao, this)
    }

    def incorporarFuncao( funcaoAdaptativa ) {
        tabFuncoes.incorporarFuncao(funcaoAdaptativa)
    }

    def gerarPrograma( nomePrg ) {
        def programa="""\
// GERADO AUTOMATICAMENTE PELO gerador.Uba
package demo

class $nome extends uba.UbaAbs {

    def iniciar() {\n\
        estados = ${tabEstados.estados()}
        inicial = ${tabEstados.inicial()}
        finais = ${tabEstados.finais()}
        regras = ${tabRegrasAdaptativas.regrasAdaptativas()}
        super.iniciar()
    }

    ${tabFuncoes.funcoes()}
}
        """.stripMargin()

        print("Gerando $nomePrg ...")
        def arq = new File(nomePrg)
        arq.write( programa )
        println( "pronto!")
    }
}
\end{lstlisting}



