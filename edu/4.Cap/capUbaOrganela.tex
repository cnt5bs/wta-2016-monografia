\chapter{Tradução gênica: UBA inspirada em comportamento genético (UBA-TG)}

Em genética, a proteína representa o fenótipo de um indivíduo ou de uma espécie, ou seja, seu metabolismo e comportamento. O ácido desoxirribonucleico (DNA), por sua vez, representa o genótipo, conjunto de informações responsáveis pela replicação e procriação. O ácido ribonucleico (RNA) é a substância química que liga os dois mundos: informação e expressão. Finalmente, tradução gênica é o nome dado ao processo que, partindo da informação genética, fabrica as proteínas necessárias à vida~\cite{ridley99}.

Na tradução gênica, uma fita de RNA mensageiro (RNAm) é produzida no núcleo celular e migra ao citoplasma onde é acolhida pelo ribossomo. O ribossomo é uma estrutura bipartida composta de RNA ribossômico (RNAr) que atua como uma máquina de tradução/fabricação. Primeiro, o ribossomo encontra o início da mensagem na fita de RNAm, então convoca uma outra substância também presente no citoplasma,o RNA transportador (RNAt), que será responsável por levar até o ribossomo o aminoácido que inicia a tradução da mensagem do gene. Após o início do processo, uma cadeia de aminoácidos é formada pela convocação de outros RNAt que transportam  moléculas de aminoácidos específicas, segundo a especificação da fita de RNAm~\cite{ridley99}.

\section{Modelagem semiótica da UBA-TG: passos 1 ao 5 do MTS}

\subsection{Passo 1: análise preliminar}

O trabalho da máquina ribossômica de tradução/fabricação ocorre de maneira precisa e rápida. Percebe a presença de RNAm e, num primeiro momento, trabalha com o objetivo de encontrar o início da mensagem. Num segundo momento, ajusta seu comportamento para a tradução/fabricação propriamente dita.

A tradução gênica, com sua dinâmica e seus dois comportamentos, reconhecimento e tradução/fabricação, aponta para aplicações potenciais em diversos campos como os da criptografia, segurança e privacidade de redes e reconhecimento de linguagem natural entre outros.


\subsection{Passo 2: definição da arquitetura de subordinação}

Duas camadas de subordinação podem ser abstraídas do fenômeno de tradução gênica. Tendo em vista o comportamento do ribossomo, uma primeira camada refere-se às atividade de percepção e busca. Nesta camada, o ribossomo permanece em constante estado de atenção até perceber a presença de uma nova fita a ser traduzida. Ao percebê-la, busca o início da mensagem.

A segunda camada é responsável pela tradução/fabricação propriamente dita. A camada de busca seria hierarquicamente superior à camada de percepção, pois, iniciado o processo de tradução/fabricação, ele não será interrompido pela presença de nenhuma nova fita até que o ribossomo encontre o final da mensagem. %(Figura 3.1).

\figLatHere{\eduPastaFig}{ubatgsubordinacao}{Arquitetura de subordinação abstraída do fenômeno de tradução gênica.}

\subsection{Passo 3: definição do nível focal}

O nível focal adequado ao estudo da transposição semiótica do fenômeno de tradução gênica é o nível das organelas celulares. Consequentemente, o nível inferior ou micro-semiótico, iniciador dos processos, é o nível genético das moléculas de RNA e o nível superior ou macro-semiótico, que apresenta as restrições naturais, é o nível celular %(Figura 3.2).

\figLatHere[0.6]{\eduPastaFig}{ubatgnivelfocal}{Representação dos níveis hierárquicos abstraídos do fenômeno de tradução gênica.}

\subsection{Passo 4: levantamento das semioses relevantes}

Em genética, códon é uma sequência de três bases hidrogenadas. No RNA essas bases podem ser Adenina (A), Guanina (G), Uracila (U) e Citosina (C). Uma fita de RNAm contém uma sequência de códons a ser traduzida pelo ribossomo. O ribossomo, dividido em duas subpartes, move-se ao longo da fita de RNAm traduzindo cada códon encontrado em um aminoácido específico dentre vinte verificáveis. Ao ler certo códon, o ribossomo convoca o RNAt capaz de se acoplar a ele através de seu anticódon, sendo que, os acoplamentos possíveis são A com U e G com C. Cada RNAt convocado transporta um aminoácido específico. Após a leitura de todos os códons, forma-se uma proteína, ou seja, uma cadeia de aminoácidos dobrada em uma forma distinta de acordo com sua sequência.

Dentre os vinte  códons, a sequência AUG indica o início da mensagem a ser traduzida, e as sequências UAA, UAG e UGA indicam o final da mensagem. Assim, o códon inicial (AUG) só poderá se acoplar a um RNAt que contenha o anticódon UAC~\cite{ridley99,lodish12}.

\subsubsection*{Camada de percepção e busca}

Esta camada comportamental pode ser representada heuristicamente por duas semioses relevantes: S1, S2. S1 seria a semiose de percepção da fita de RNAm e a S2 a semiose de busca pelo início da mensagem. Enquanto S1 ocorre apenas uma vez, S2 se repete até que o início seja encontrado. Ambas semioses ocorrem na subparte menor do ribossomo.

\begin{itemize}
	\item \textbf{Semiose 1 - Percepção da fita de RNAm.}

		Mediante a necessidade celular em fabricar certa proteína, a presença de uma fita de RNAm provoca o início do processo de tradução gênica. Neste instante, ocorre a conexão da subparte menor do ribossomo com o sítio de ligação da fita de RNAm (\textit{Ribossome Bindind Site - RBS}). Após essa ligação, a subparte menor do ribossomo acopla-se a uma molécula de RNAt que carrega uma unidade do aminoácido metionina. O RNAt convocado deve apresentar códon AUG, necessário para acoplamento ao códon UAC da fita de RNAm que é indicativo do início da mensagem.

		Em termos semióticos (tríade Objeto/Signo/Interpretante) tem-se: a fita de RNAm como o sinal perceptivo que designa o objeto (O), a ligação RBS/Ribossomo como signo (S) e o acoplamento entre a subparte menor do ribossomo e a molécula de RNAt (AUG) como interpretante (I).

		\figLatHere[0.6]{\eduPastaFig}{ubatgs1-crop}{Características da semiose S1.}

		\item \textbf{Semiose 2 - Busca do início da mensagem.}

		Após o acoplamento entre a subparte menor do ribossomo e o RNAt (AUG), o ribossomo se desloca para o primeiro códon, então dois resultados são possíveis: se este códon for do tipo UAC, ele se acopla ao RNAt (AUG) produzindo a metionina na subparte maior do ribossomo e o sistema passa para a semiose S3 na camada de tradução/fabricação, caso contrário, o ribossomo se desloca para o códon seguinte realizando nova S2, permanecendo na camada de percepção e busca até encontrar o início da mensagem.

		Em termos semióticos (tríade Objeto/Signo/Interpretante) tem-se: o acoplamento ribossomo/RNAt (AUG) como o sinal perceptivo que designa o objeto (O), o posicionamento sobre o códon de leitura como signo (S) e o resultado da ação (acoplamento ou não) como interpretante (I).

		\figLatHere[0.6]{\eduPastaFig}{ubatgs2-crop}{Características da semiose S2.}

\end{itemize}

\subsubsection*{Camada de tradução/produção}

Esta camada comportamental também pode ser representada por duas semioses: S3 é responsável pela leitura dos códons seguintes e consequente encadeamento dos aminoácidos que formarão a proteína requisitada (S3 ocorre de forma recursiva até o códon final), e S4 é responsável pelo fechamento da cadeia de aminoácidos.

\begin{itemize}
	\item \textbf{Semiose 3 - Formação da cadeia de aminoácidos}

	Após a identificação do início da mensagem contida na fita de RNAm e a consequente produção da metanina, o ribossomo se posiciona sobre o próximo códon e produz o aminoácido correspondente. O ribossomo reproduz esse comportamento de forma recursiva até que encontre um dos seguintes códons de fechamento: UAA, UAG ou UGA.

	Em termos semióticos (tríade Objeto/Signo/Interpretante) tem-se: a metionina como sinal perceptivo que designa o objeto (O), o posicionamento sobre o próximo códon de leitura como signo (S) e a fabricação de uma nova enzima como (I).

	\figLatHere[0.6]{\eduPastaFig}{ubatgs3-crop}{Características da semiose S3.}

	\item \textbf{Semiose 4 - Fechamento da cadeia enzimática}

	Ao encontrar um dos códons de fechamento (UAA, UAG ou UGA) o ribossomo desliga a fabricação de proteína sem agregar nenhum novo aminácido.

	Em termos semióticos (tríade Objeto/Signo/Interpretante) tem-se: o último aminoácido domo o objeto (O), o posicionamento sobre o próximo códon de fechamento como signo (S) e a finalização do processo como interpretante (I).

	\figLatHere[0.6]{\eduPastaFig}{ubatgs4-crop}{Características da semiose S4.}

\end{itemize}

\subsection{Passo 5: modelagem semiótica}

Para que uma célula possa fabricar as proteínas necessárias ao organismo, o RMAm formado no núcleo da célula migra para o citoplasma. O encontro da fita de RNAm com um ribossomo inicia o processo de tradução gênica que vai sintetizar proteínas específicas segundo o código contido na fita de RNAm.

\paragraph*{Camada de percepção e busca}

O ribossomo encontra a fita de RNAm (objeto de S1) conectando sua sub-parte menor ao sítio de ligação RBS (signo de S1). Devido às características intrínsecas ao fenômeno, o resultado deste encontro é a conexão do RNAt que transporta o aminoácido metionina (interpretante de S1 e objeto de S2). O sistema inicia a procura pelo início da mensagem posisionado-se sobre o primeiro códon (signo de S2). Há dois interpretantes possíveis para S2: se o primeiro códon for UAC existe a conexão com a inserção do primeiro aminoácido (metionina) (objeto de S3) e a cadeia semiótica avança para S3, caso contrário, o sistema se posiciona sobre o segundo códon e S2 se repete e, continua a se repetir códon após códon até que ocorra a conexão AUG/UAC.

\paragraph*{Camada de tradução/produção}

O encontro do início da mensagem com a conexão AUG/UAC e a inserção da metionina (objeto de S3) tem como signo o posicionamento sobre o códon seguinte (signo de S3) e a consequente inserção do aminoácido correspondente (interpretante de S3). A semiose S3 se repete com objetos específicos segundo os códons encontrados, formando a cadeia de aminoácidos. A produção de certo aminoácido (objeto de S4) faz com que o sistema se posicione sobre um códon do tipo UAA, UAG ou UGA (signo de S4) que indica o final do processo. O interpretante de S4 é a finalização da tradução/fabricação.

\figLatTop[0.8]{\eduPastaFig}{ubatgmodelosemiotico-crop.eps}{Diagrama representando o modelo semiótico abstraído do comportamento tradução gênica.}

\section{Do modelo semiótico à UBA-TG: passo 6 do MTS}

Codificação da UBA-TG usando a linguagem de especificação UBA.

\section{Aplicação (didática): atendimento do Poupa Tempo }

\subsection{O problema}

Apresentar o problema

\subsection{Aplicação}
