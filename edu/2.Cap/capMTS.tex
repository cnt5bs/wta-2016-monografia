\chapter{Método de Transposição Semiótica}

O Método de Transposição semiótica (MTS) foi concebido como ferramenta para auxiliar a programação de computadores através de uma abordagem biomimética (Computação Bioinspirada). O programador que se utiliza de tal método busca inspiração nos comportamentos de estruturas biológicas para identificar possíveis caminhos de solução para problemas computacionais. Antes que sejam especificados os passos que integram o MTS, cabe delimitar o escopo de sua aplicação.

Uma primeira questão diz respeito à possibilidade de reconhecimento de propriedades isomórficas entre comportamentos biológicos e computacionais. Aqui, isomorfismo é entendido como definido por Hofstadter~\cite{hofstadter79}, ou seja, como a propriedade que permite duas estruturas complexas serem mapeadas entre si, de tal maneira que, para cada parte de uma dessas estruturas, exista uma parte correspondente na outra estrutura, com ambas desempenhando papéis similares. Devido à maior complexidade do campo biológico, fica clara a impossibilidade de isomorfismo completo. No entanto, o estabelecimento de alguns parâmetros restritivos pode garantir, ao menos, uma correspondência parcial capaz de mapear os elementos essenciais do fenômeno biológico em funções computacionais. Ou seja, somente os comportamentos biológicos passíveis de serem traduzidos (ou interpretados) em funções computáveis devem ser considerados. A hipótese principal do MTS é de que a semiótica aplicada pode auxiliar o programador em suas escolhas heurísticas.

Semiótica pode ser entendida como a ciência dos signos e dos processos significativos (semioses) na natureza e na cultura. A semiótica parte de um paradigma segundo o qual os atos interpretativos são compostos por encadeamento de signos. A semiótica de Peirce~\cite{peirce08}, de caráter geral, apresenta o signo como uma tríade composta por objeto, signo e interpretante. O objeto consiste em uma substância concreta do mundo fenomênico e divide-se em objeto dinâmico (o objeto em sua realidade mesma) e objeto imediato (suas qualidades perceptivas). O objeto é percebido, por certa entidade capaz de apreender suas qualidades parciais através de um signo, ou representação (ou \textit{representamen}). O efeito do signo nesta entidade produz uma interpretação sobre o objeto e sua representação, este é o interpretante. Assim, um signo peirceano é composto por um objeto, sua representação e seu interpretante. A esta inter-relação ou processo é dado o nome de semiose~\cite{peirce1940philosophical}.

Uma cadeia semiótica se instala quando o interpretante de uma semiose original torna-se o objeto de um novo signo. Isto pode se repetir indefinidamente e cada semiose aumenta a abstração ao afastar o interpretante do objeto real presente no mundo fenomênico. No entanto, um mesmo objeto real (objeto dinâmico) pode dar início a cadeias semióticas distintas já que os vários aspectos de sua realidade (objeto imediato) geram diferentes representações e interpretações de acordo com as disposições perceptivas e cognitivas da entidade afetada pelo signo. Ressalta-se que essas diferenças interpretativas podem se manifestar em qualquer ponto da cadeia semiótica, formando assim nuanças de signo.

Há de se entender que, na aplicação do MTS, vários níveis de processos semióticos estão em jogo. Primeiro, existem as semioses reais (objeto real) inerentes a cada estrutura biológica, cujas interpretações são vivenciadas apenas por elas mesmas, não sendo possível ao observador acessá-las diretamente. No entanto, o observador, através de um processo semiótico próprio, é capaz de interpretar o fenômeno (por efeito do objeto imediato) e estabelecer uma cadeia semiótica virtual, um modelo abstrato do fenômeno. Finalmente, o observador utiliza a cadeia semiótica virtual como objeto de uma nova semiose para estabelecer a correspondência computacional. Em resumo, os seguintes níveis semióticos estão presentes: (1) a cadeia semiótica inerente ao fenômeno e não acessível diretamente ao observador. O agente interpretador desta cadeia semiótica é a própria estrutura biológica afetada por ela; (2) a cadeia semiótica que se instala no observador/programador em seu ato interpretativo. O agente interpretador é o próprio observador; (3) a cadeia semiótica virtual como modelo abstrato. O agente interpretador desta cadeia semiótica seria uma instância hipotética da estrutura biológica; e (4) a cadeia semiótica responsável pela interpretação da cadeia semiótica virtual em computação. Aqui, o agente interpretador é o observador/programador. O MTS atua diretamente nas semioses do nível 2, acima, auxiliando o estabelecimento dos níveis 1 e 4.

Para estudar a computação por meio do paradigma semiótico é necessário expressá-la por meio de signos.

%Desde aqui
%Pensar junto com o Ítalo e Francisco. não seria objeto = fita na entrada, signo = estado Q correspondente e interpretante = fita na saída?

Entende-se computação como cálculo efetivo e um modelo que o realiza é a Máquina de Turing. Grosso modo, a Máquina de Turing funciona por meio de um conjunto de transições parciais que descrevem um cálculo, isto recebe o nome de \emph{especificação}. A realização da especificação faz com que a Máquina de Turing assuma determinado \emph{comportamento}, o qual, ao modificar a fita traduz-se em \emph{funcionamento}. Portanto, \emph{funcionamento} é a percepção do operador ao observar mudanças na fita; \emph{comportamento} são as ações de movimentação, leitura e escrita efetuadas pela máquina ao realizar a \emph{especificação} \cite{turing36}.

Neste sentido, a Máquina de Turing pode ser expressa em termos semióticos pela tríade especificação/comportamento/funcionamento. Tal qual ocorre com um signo peirceano a especificação é algo concreto, estável e identificável. Por outro lado o comportamento que realiza a especificação, tal qual a representação de um objeto, é variável. O funcionamento, por sua vez, produz na mente do operador um interpretante o qual será utilizado como objeto para a realização de semioses sobre o programa. Daí se afirmar que o signo computacional (ou signo de Turing) ser composto por especificação, comportamento e funcionamento. Cabe lembrar que o signo é necessariamente uma construção mental e portanto este é o signo que se costuma formar na mente ao compreender a Máquina de Turing.

Aplicando recursivamente a análise semiótica sobre o comportamento tem-se que a Máquina~de~Turing é um instrumento para o cálculo de números computáveis, onde, para Turing um número é computável se puder ser escrito por uma máquina em termos finitos \cite{turing36} ou por uma sobreposição de intervalos \cite{turing36_errata}. Uma vez que se está propondo usar a semiótica na computação convém explorar a ideia de signo computável. Em termos peirceanos um número pode ser classificado como legi-signo \cite{winfried}, assim, um legi-signo é computável se o \textit{representamen} puder ser escrito por uma Máquina de Turing em termos finitos.

Existem seis classes de legi-signos \textit{cf.}~\cite{teoriaSignos_Santaella}: 1)  o icônico-remático; 2) indicativo-remático; 3) indicativo-dicente; 4) simbólico-remático; 5) simbólico-dicente; e 6) simbólico-argumental. Um legi-signo é icônico quando o interpretante representa uma similaridade; indicativo quando o interpretante representa um ponteiro; e simbólico quando o interpretante representa hábitos de uso. Um legi-signo é remático quando o interpretante representa uma possibilidade; dicente quando o interpretante representa um fato; e argumento quando o interpretante representa uma lei.

Em computação o legi-signo assume a forma de um \emph{tipo abstrato} onde o objeto é uma referência à um objeto no domínio do problema, o \textit{representamen} um simbolo computável e o interpretante é definido \textit{a priori} pelas operações do tipo (ou funções de transição da Máquina de Turing). Isto, por sua vez propicia as quasi-semioses \cite{winfried}, isto é, semioses definidas \textit{a priori} e de cunho automático; um paralelo desta idéia na computação são comportamentos polimórficos ou adaptativos.

% até aqui

Assim, existem dois níveis de interpretação semiótica na Máquina de Turing. O primeiro nível consiste na formação do signo na mente do \emph{operador} e o segundo nível a forma pela qual os signos foram desenvolvidos e realizados na máquina pelo \emph{programador}.

Considerando que é possível expressar a Máquina de Turing em termos semióticos e dado que o mesmo é possível para organismos vivos e meio-ambiente como a zoossemiótica, biossemiótica e os processos microbiológicos, genéticos e evolutivos~\cite{noth95, noth96}. É possível sugerir que o campo da semiótica possa ser utilizado como campo intermediário na transposição dos aspectos essenciais de um determinado fenômeno biológico do campos natural ao campo computacional, agindo assim como elemento heurístico.

Partindo desta concepção, propõe-se neste estudo uma heurística de transposição semiótica intitulada Método de Transposição Semiótica (MTS). O MTS fundamenta-se em três corpos teóricos, dois deles de base semiótica, a teoria da significação de Uexküll~\cite{uexkull34, uexkull82} e o estruturalismo hierárquico de Salthe~\cite{salthe85}; e o outro, baseado na inteligência artificial, mais precisamente, na arquitetura de subordinação de Brooks~\cite{brooks99a, brooks99b, brooks99c}.

\section{Arquitetura de subordinação de Brooks}

Brooks desenvolveu na década de 1980 uma crítica consistente ao modelo de IA clássica, marcadamente baseada no representacionalismo simbólico e sua estratégia \textit{top-down}. Para ele, a IA clássica, ao partir do pressuposto de que cognição e representação seriam a mesma coisa, passa a concentrar esforços na simulação de atividades cognitivas superiores para, então, tentar simular atividades mais básicas. Em termos peirceanos, Brooks considera que a terceiridade, enquanto representação de mundo, não deva ser dada \textit{a priori} mas alcançada pelo processo normal de construção do signo. Ou seja, que o automato, por meio de sensores, deve se utilizar das possibilidades de encontros com os objetos do mundo (primeiridade) para ajustar seu conjunto de regras (terceiridade), e isto acarreta um comportamento (secundidade) adequado ao contexto específico no qual ele está inserido. Portanto, o que Brooks sugere é que a programação deva ter foco na secundidade e não na terceiridade.

Na terminologia de \textit{problem-frames} \textit{cf.}~\cite{Jackson:2000:PFA:513720} significa que não deve haver um modelo \textit{a priori} que represente o mundo do problema e que a máquina deve apresentar um comportamento polimórfico-adaptativo capaz de se adequar ao mundo do problema. Ao construir seus robôs, Brooks e sua equipe chegam à conclusão de que quando são examinados níveis muito simples de inteligência, percebe-se que as representações explícitas e modelos de mundo apenas atrapalham, e levantam a hipótese de que a representação do mundo do problema seria uma unidade errada de abstração ao construir a maior parte dos \emph{sistemas inteligentes}~\cite{brooks99a}.

Para Brooks, a história da evolução biológica desde as células simples até a inteligência humana sugere que as capacidades superiores humanas como linguagem, razão e capacidade de solução de problemas complexos só foram possíveis após a evolução ter proporcionado soluções reativas para problemas menores como a capacidade de organismos primitivos se moverem num ambiente dinâmico, sentindo este ambiente e agindo no sentido da autopreservação e reprodução. Isto justifica o  estudo da inteligência segundo uma estratégia \textit{bottom-up} focada em sistemas físicos (robôs móveis, por exemplo) situados num mundo e autonomamente cumprindo tarefas variadas. Alguns desses trabalhos são baseados em princípios da engenharia e outros são firmemente baseados em inspirações biológicas~\cite{brooks99a, brooks99b, brooks99c}. Em suma, Brooks argumenta que os projetos baseados em circuitos são tudo o que é necessário à IA, que a representação e o raciocínio são incômodos, dispendiosos e desnecessários.

Este é o aspecto conceitual da Arquitetura de Subordinação~\cite{brooks99a, brooks99b, brooks99c}. Ainda para Brooks, esta concepção é realizada computacionalmente por meio do encadeamento de camadas compostas por máquinas de estados finitos. Ou seja, ele sustenta que todo trabalho de engenharia tem a necessidade de decompor um sistema complexo em partes menores, ou módulos, para construir cada parte e então integrá-las num sistema completo. Para proceder a esta modularização, existem duas estratégias básicas: modularização por função ou modularização por atividade. Na construção de seus robôs, Brooks opta pela segunda. Com esta estratégia Brooks consegue conectar cada \textit{input} diretamente a uma saída gerando camadas comportamentais \textit{cf.}~\cite{parnas}.

A vantagem deste método é que proporciona uma construção incremental, das atividades mais simples às mais complexas. Por exemplo, se for necessário fazer uma alteração no módulo \textit{andar}, não ocorrem mudanças indesejadas em outros módulos. Às vezes esta abordagem é chamada de comportamentalista e suas premissas são: \textbf{objetos situados}, os robôs são situados num mundo, não lidando com descrições abstratas mas com o aqui e agora deste mundo; \textbf{objetos corporificados}, os robôs possuem um corpo e experienciam o mundo através de seus sensores e atuadores; \textbf{inteligência}, os robôs são percebidos como inteligentes, mas isto não deriva simplesmente de sua capacidade computacional mas também do acoplamento com o mundo; e \textbf{emergência}, a inteligência do sistema emerge da interação do sistema com o mundo e às vezes de interações indiretas de seus componentes. Portanto, a nova IA de Brooks baseia-se na hipótese da fundamentação física que estabelece que para se construir um sistema inteligente é necessário ter suas representações  fundamentadas no mundo físico.

\section{Estruturalismo Hierárquico de Salthe}

O estruturalismo hierárquico, tal qual é abordado na obra de Salthe~\cite{salthe85}, é um estudo de meta-teoria científica a partir da perspectiva da biologia e pretende lidar com a representação das coisas do mundo e suas relações. É uma teoria sistêmica que, apesar de originar-se nas ciências biológicas, sua aplicação pode ser útil em qualquer outro campo do conhecimento.

Para El-Hani e Queiroz~\cite{queiroz07}, o trabalho de Salthe é uma proposta de abordagem de sistemas complexos que apresenta o sistema triádico básico de influência peirceana como elemento fundamental. Segundo o estruturalismo hierárquico, para se descreverem as interações e processos entre entidades fundamentais num dado sistema precisa-se: primeiro, identificar em  qual nível focal o processo se instala; segundo, investigar tal processo em termos das entidades no nível anterior; e, terceiro, considerar as entidades que operam num nível superior. Para processos semióticos, consideram-se os níveis como: focal, micro-semiótico e macro-semiótico. O nível micro-semiótico apresenta todas as tríades possíveis a partir de um mesmo objeto dinâmico, assim, certa semiose que surge no nível focal seria a efetivação do conjunto de algumas possibilidades armazenadas no nível micro-semiótico; o nível macro-semiótico, por sua vez, representa o conjunto de semioses possíveis a um determinado sistema funcionando como elemento limitador.

Complementando, o nível inferior, micro-semiótico ou micro-sistêmico é o único do qual emana a capacidades criativa do sistema. As possibilidades que devem ocorrer no nível focal são geradas a partir deste nível e só algumas delas serão realizadas quando confrontadas com as restrições do nível superior~\cite{salthe85}.

No estruturalismo hierárquico, as noções de \textbf{objeto}, \textbf{entidade} e \textbf{coisa} estão interligadas. Assim, objetos são experienciados implicando a presença de um sujeito, as coisas são acreditadas e as entidades definidas. Para que uma coisa possa ser considerada uma entidade, deve apresentar uma \textbf{fronteira} discernível, deve ser um sistema cibernético bem \textbf{integrado} e, finalmente, deve apresentar \textbf{continuidade espaço-temporal}. Assim, uma entidade se define como algo de certo tamanho, distinto de sua cercania, possuindo sub-partes em relações cibernéticas e estabilidade ao longo do tempo~\cite{salthe85}.

Quando um determinado fenômeno é analisado sob a ótica do estruturalismo hierárquico, deve-se proceder à escolha do nível focal segundo as necessidades e interesses do observador. Um quadro completo de níveis hierárquicos pode ser descrito através dos seguintes níveis: atômico, molecular, organelas intracelulares, células, tecidos, órgãos, aparelhos, organismo, nicho ecológico, ecossistema, meio-ambiente, de acordo com necessidades específicas alguns desses níveis podem ser condensados e outros suprimidos, por exemplo, uma estrutura possível seria considerar como nível focal o nível das células e suas organelas. Neste caso, pode-se entender o nível molecular como o nível inferior e o nível do organismo como nível superior, suprimindo os níveis dos tecido, órgãos e aparelhos~\cite{salthe85}.

\section{A teoria da significação de Uexküll}

A teoria biossemiótica de Jakob von Uexküll, ou teoria da significação biológica, apresenta-se como uma possibilidade de grande alcance para o estudo das relações entre animais e o meio-ambiente. Esta abordagem, segundo Thure von Uexküll~\cite{thure04}, adota o ponto de vista sistêmico, negando tanto o objetivismo positivista quanto o subjetivismo idealista e aponta para a interação entre sujeito e objeto como  uma inter-relação em um todo maior.  O  objeto não pode ser estudado separadamente de seu meio-ambiente. Uexküll cria, então, o conceito de \textit{Umwelt} que seria o segmento ambiental de um organismo, definido por suas capacidades específicas da espécie, tanto receptoras quanto efetoras. Segundo Uexküll:

\begin{quote}\textit{Todos os animais, do mais simples ao mais complexo, ajustam-se dentro de seus mundos únicos com igual completude. Um mundo simples corresponde a um animal simples, um mundo bem-articulado a um animal complexo.}~\cite{uexkull34} \end{quote}

Para Uexküll~\cite{uexkull34, uexkull82}, as relações entre sujeito e objeto que definem o \textit{Umwelt} de uma determinada espécie são representadas através de um diagrama que ele denomina círculo funcional e que caracteriza a dinâmica bem ajustada entre o objeto (\textit{meaning-carrier} ou portador da significação) e o sujeito (\textit{meaning-receiver} ou receptor da significação). O objeto é portador de dois tipos de pistas referentes ao processo de significação que está em jogo, \textbf{pista perceptiva} e \textbf{pista operacional}. Através da primeira, o objeto é percebido pelo  animal ao captar em seu órgão receptor um determinado \textbf{sinal perceptivo}. Segundo as características da sua espécie, o órgão receptor inicia a transmissão de impulsos correspondentes em direção ao \textbf{órgão operacional}. O próprio objeto fornece a \textbf{pista operacional} através do \textbf{sinal operacional} que extingue a pista perceptiva completando o círculo funcional. Os sinais provenientes do objeto são inerentes ao objeto e as pistas que projeta são inerentes às capacidades de percepção do animal, ou seja, sinais são traduzidos em pistas, segundo a característica de cada espécie. Ainda, em outras palavras, a interação com um objeto causa no animal a formação de um signo. Como exemplo: o ácido butírico exalado pelas glândulas dos mamíferos (sinal) seria captado pelo órgão receptor  de outro animal qualquer e percebido como estímulo olfativo (pista) (Fig~\ref{0203circulofuncional-eps-converted-to}).

\figLatTop[0.8]{\eduPastaFig}{0203circulofuncional-eps-converted-to}{Circulo funcional de Uexküll.}

No modelo de Uexküll, estruturas mais simples como células também apresentam características de \textit{Umwelt}. O \textit{Myxomyces}, um tipo de fungo, apresenta seus esporos em estado inicial de desenvolvimento como células parecidas com amebas com movimentos livres, alimentando-se de bactéria floral. Nas palavras de Uexküll:

\begin{quote}\textit{Somos forçados a atribuir \textit{Umwelt}, mesmo que limitado, às células-fúngicas livres e vivas, um \textit{Umwelt} comum a cada uma delas, no qual a bactéria  contrasta com o entorno, como portadoras de significação. como alimento e, assim, são percebidas causando reação. Por outro lado, o fungo, composto de muitas células unitárias, é uma planta que não apresenta \textit{Umwelt}t animal - é apenas cercado por um tegumento-habitável constituído de fatores de significação}~\cite{uexkull34} \end{quote}

Os vegetais, segundo Thure von Uexküll~\cite{thure04}, como apenas tegumentos habitáveis, não detectam qualidades especiais nos sinais perceptivos e operacionais, portanto tais sinais não geram pistas e se mostram suficientes para a realização de processos sígnicos.  Neste caso, não seria o círculo funcional a descrever as semioses, mas sim o sistema retroativo \textit{feedback system}. Os signos (sinais) perceptivos são codificados por um receptor e os signos operacionais, através da atividade de um efetor, ajustam o valor real de um sistema variável (o tegumento habitável de um vegetal ou célula), fazendo este valor concordar com o valor referencial requerido.

Desta forma, os processos semióticos distribuem-se segundo níveis hierárquicos. Para Thure von Uexkül~\cite{thure92}, essa hierarquia pode ser expressa em termos do usuário da significação (\textit{meaning-utilizer}): sistemas de signos intracelulares, no qual as organelas seriam  \textit{meaning-utilizers}; sistemas de signos intercelulares e celulares, que corresponderia à endossemiótica; sistemas de relações sígnicas entre animais ou zoossemiótica e sistemas sígnicos entre grupos sociais através da linguagem humana. Para ele, o sistema de Jakob von Uexküll estaria confinado ao segundo e terceiro desses tipos de sistemas. Para Sebeok \cite{sebeok99}, no entanto, a endossemiótica também teria em seu campo de estudo as relações sígnicas entre organelas celulares, células, tecidos, órgãos, além de relações genéticas. Assim, a teoria da significação de Uexküll, segundo Ziemke~\cite{ziemke01}, pode ser de grande ajuda no entendimento do uso de signos e representações pelos organismos vivos e na compreensão das possibilidades e limitações de autonomia e semiose em organismos artificiais~\cite{ziemke01}.

\section{Premissas do MTS}

O estudo de um determinado fenômeno biológico origina, através do MTS, um meta-modelo computacional que representa as semioses essenciais subjacentes a tal fenômeno. Para a construção deste meta-modelo, o MTS utiliza a arquitetura de subordinação de Brooks para determinar as camadas responsáveis por comportamentos individuais. A composição desses comportamentos individuais dá origem ao comportamento global do sistema sendo que camadas superiores prevalecem sobre camadas inferiores (Fig~\ref{0205subordinacao-eps-converted-to}). Como critério de decomposição das camadas foi utilizada a decomposição por responsabilidades proposto por Parnas em \cite{parnas72}. Desta forma, tendo em vista a implementação computacional final, são identificados com base no animal os possíveis módulos biológicos que podem ser implementados em módulos computacionais.

\figLatHere[0.6]{\eduPastaFig}{0205subordinacao-eps-converted-to}{Arquitetura de subordinação baseada no trabalho de Brooks.}

O tratamento da complexidade inerente aos sistemas biológicos requer um mecanismo capaz de auxiliar a pesquisa no encontro das entidades biológicas relevantes à abstração de um algoritmo biológico. Para tanto, o MTS utiliza o estruturalismo hierárquico de Salthe. Para Salthe, deve-se proceder à escolha do nível focal segundo as necessidades e interesses do observador, consequentemente, os níveis inferior e superior se estabelecem. O nível focal é onde se realiza a cadeia semiótica correspondente a um determinado fenômeno; o inferior é o nível das possibilidades ou nível iniciador onde se encontram as semioses potenciais; e o nível superior é o nível das restrições segundo as características de uma determinada espécie~\cite{salthe85} (Fig~\ref{0206estruturalismohierarquico-eps-converted-to}).

\figLatHere[0.6]{\eduPastaFig}{0206estruturalismohierarquico-eps-converted-to}{Estrutura hierárquica esquemática.}

A arquitetura de subordinação e o estruturalismo hierárquico atuam como elementos organizadores, sendo que o último, através do conceito de nível focal, também auxilia no direcionamento do olhar do pesquisador em busca das entidades relevantes ao algoritmo biológico em questão. Mas, quais seriam essas entidades?  Adota-se, aqui que uma entidade relevante é aquela responsável por uma semiose. Mas, existem semioses no contexto biológico? Para responder a esta questão a pesquisa adota a teoria da significação de Uexküll como corpo teórico que sustenta a resposta positiva. Adota também o pressuposto de Sebeok~\cite{sebeok99} de que processos biológicos podem ser vistos como processos triádicos, mais adequado ao estudo de semioses em nível celular, ultrapassando, assim, o conceito de círculo funcional de Uexküll.

Finalmente, como representação gráfica de uma semiose triádica adota-se o \textit{tripod} que, segundo Queiroz~\cite{queiroz04}, é a estrutura que melhor representa esta relação, pois o triângulo é, na verdade uma relação de pares de termos sendo que nenhuma combinação de vértices produz uma relação triádica (figura~\ref{0207tripod-eps-converted-to})

\figLatHere[0.6]{\eduPastaFig}{0207tripod-eps-converted-to}{O \textit{tripod} como representação gráfica de uma semiose.}

Essas passagens são descritas graficamente na Fig~\ref{0208conjunto-eps-converted-to}. Onde, partindo-se de um determinado fenômeno biológico (A), são identificadas, no nível focal e de acordo com uma arquitetura de subordinação adequada, as semioses relevantes à abstração do algoritmo biológico dos comportamentos em estudo, esses elementos formam um modelo semiótico que representa o algoritmo em questão (B); cada semiose representada no modelo semiótico é transposta como uma entidade computacional gerando um meta-modelo computacional (C) finalizando a transposição e concluindo o MTS. A partir do meta-modelo, várias aplicações podem ser derivadas dando origem a dispositivos computacionais específicos a diferentes domínios (D).

\figLatTop[0.6]{\eduPastaFig}{0208conjunto-eps-converted-to}{Método de Transposição Semiótica - diagrama geral.}

Com base nisso o MTS considera o \textit{tripod} como uma representação da estrutura de um \emph{tipo abstrato} e seu comportamento é determinado por meio de uma máquina de estados finitos as quais são encadeadas para formarem o construto computacional. Como o polimorfismo é um elemento central das construções biomiméticas optou-se como tecnologia os autômatos adaptativos devido a possuir poder de expressão conveniente para este tipo de sistema \cite{italo}.
