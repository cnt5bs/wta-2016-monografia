Esta monografia foi desenvolvida sob a luz da multidisciplinaridade, reunindo articulações entre três áreas de conhecimento, a biologia, a computação e, como elemento de ligação, a semiótica, dando origem à convergência ente computação bioinspirada e a teoria dos autômatos adaptativos.

O \textbf{Método de Transposição Semiótica} - fundamentado em outros três corpos teóricos, dois deles de base semiótica, a teoria da significação de Uexküll e o estruturalismo hierárquico de Salthe, e o outro, baseado na nova IA, mais precisamente, na arquitetura de subordinação de Brooks - foi utilizado como elemento de ligação entre os campos biológico e computacional. Este método, apresenta seis passos que o estabelecem como um método geral. Esses passos, em seus devidos momentos, estabeleceram técnicas especiais baseadas em recortes teóricos específicos extraídos dos trabalhos de Uexküll, Salthe e Brooks.

A escolha dos comportamentos de habituação e sensibilização da aplísia, apresentados por Kandel, mostraram-se bastante adequados como estudo de caso, primeiramente, pela precisão e clareza com que Kandel descreve suas descobertas no campo da neurociência e, segundo, por proporcionar um recorte de dimensões tratáveis no tempo relativamente curto da pesquisa.

Tendo em vista que a proposta do MTS é de abstrair somente os mecanismos essenciais subjacentes ao algoritmo biológico em estudo, e não a complexidade orgânica total do animal, conclui-se que uma importante contribuição do método à modelagem computacional biomimética vem justamente de seu caráter sintético, que promove operações heurísticas auxiliando o direcionamento do olhar do pesquisador na busca dos processos sígnicos realmente imprescindíveis à transposição entre os campos biológico e computacional.

Outra contribuição a ser destacada é que o MTS se apresentou como uma ferramenta de modelagem eficaz para o seu propósito. Os modelos específicos resultantes não decorreram de simples reduções, mas uma síntese fundamentada que resultou num modelo semiótico e em seu consequente metamodelo computacional.

Além dessas duas contribuições gerais, outra de caráter mais específico pode ser descrita. Com referência ao aplicativo-exemplo, pode-se concluir que a aprendizagem baseada em habituação e sensibilização se mostrou um mecanismo interessante na implementação de buscas contextuais. Podem-se imaginar sistemas de busca em grandes volumes de dados nos quais o ajuste dos parâmetros resultem em diferentes resultados de acordo com a necessidade do usuário. Quando o tempo de busca for questão crítica, ajusta-se o sistema para tender à habituação e entregar um resultado que, mesmo não refletindo a totalidade da informação útil, consegue entregar uma parte significativa num tempo reduzido.

Além da aplicação do MTS em si, esta monografia mostrou sua MTS em conjunto com dispositivos adaptativos para a elaboração de especificações adaptativas no sentido proposto por Vega[11].

Partindo de fenômenos biológicos, por exemplo, pelo animal Aplysia, o emprego das semioses do MTS produziu um modelo comportamental que combinou estados de habituação e de sensibilização.

Estes são estados que modelam a reação do animal a diferentes tipos e intensidades de sinais por ele percebidos. Um dispositivo adaptativo foi utilizado para modelar os particulares fenômenos que desencadeiam as mudanças comportamentais do animal. A técnica de especificações adaptativas suportou a representação do modelo adaptativo assim elaborado. Um posterior refinamento originou um padrão de comportamento chamado de unidade biomimética adaptativa habituação-sensibilização (UBA-HS), também proposto nesta monografia.

Estes resultados apontam na direção de um catálogo de padrões biomiméticos adaptativos que se encontra em elaboração nesta pesquisa. Em particular, uma versão preliminar da UBA-HS foi utilizada em um ensaio de modelagem do KWIC [13] e está em estudo o seu emprego no contexto da Internet das Coisas [14], os quais não foram alvo de discussão detalhada neste trabalho.
