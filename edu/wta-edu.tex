\documentclass[12pt,graybox,envcountchap,sectrefs]{svmono}

%%% Para uso de figuras neste documento.
%%% Deve ser redefinido para gerar a monografia
\newcommand{\geralPasta}{../geral}
\newcommand{\eduPasta}{../edu}
\newcommand{\eduPastaFig}{\eduPasta/fig}


\usepackage{polyglossia}
\setmainlanguage{portuges}

% choose options for [] as required from the list
% in the Reference Guide
%\usepackage{mathptmx}
%\usepackage{helvet}
%\usepackage{courier}
\usepackage{xltxtra,fontspec,xunicode}
%\setmainfont[Ligatures=TeX]{Vollkorn}
%\setromanfont[Ligatures=TeX]{Vollkorn}
%\setsansfont[Ligatures=TeX,Scale=MatchLowercase]{DejaVu Sans}
%\setmonofont[Scale=MatchLowercase]{DejaVu Sans Mono}

\usepackage{amsmath}
\usepackage{amssymb}
\newfontfamily\listingsfont[Scale=0.7]{DejaVu Sans Mono}
\newfontfamily\listingsfontinline[Scale=0.8]{DejaVu Sans Mono}
\usepackage{listings}
	\usepackage{color}
	\definecolor{sh_comment}{rgb}{0.12, 0.38, 0.18 } %adjusted, in Eclipse: {0.25, 0.42, 0.30 } = #3F6A4D
	\definecolor{sh_keyword}{rgb}{0.37, 0.08, 0.25}  % #5F1441
	\definecolor{sh_string}{rgb}{0.06, 0.10, 0.98} % #101AF9
	\def\lstsmallmath{\leavevmode\ifmmode \scriptstyle \else  \fi}
	\def\lstsmallmathend{\leavevmode\ifmmode  \else  \fi}
	\lstset {
	language=Java,
%	frame=shadowbox,
	rulesepcolor=\color{black},
	showspaces=false,showtabs=false,tabsize=2,
%	numberstyle=\tiny,numbers=left,
	basicstyle= \listingsfont,
	stringstyle=\color{sh_string},
	keywordstyle = \color{sh_keyword}\bfseries,
	commentstyle=\color{sh_comment}\itshape,
	captionpos=b,
	xleftmargin=0.7cm, xrightmargin=0.5cm,
	lineskip=-0.3em,
	escapebegin={\lstsmallmath}, escapeend={\lstsmallmathend}
	}
	% Applies only when you use it
	\usepackage{xcolor}
	\lstdefinestyle{antlr}{
	basicstyle=\small\ttfamily\color{magenta},%
	breaklines=true,%                                      allow line breaks
	moredelim=[s][\color{green!50!black}\ttfamily]{'}{'},% single quotes in green
	moredelim=*[s][\color{black}\ttfamily]{options}{\}},%  options in black (until trailing })
	commentstyle={\color{gray}\itshape},%                  gray italics for comments
	morecomment=[l]{//},%                                  define // comment
	emph={%
	 STRING%                                            literal strings listed here
	 },emphstyle={\color{blue}\ttfamily},%              and formatted in blue
	alsoletter={:,|,;},%
	morekeywords={:,|,;},%                                 define the special characters
	keywordstyle={\color{black}},%                         and format them in black
	}
\usepackage{type1cm}
\usepackage{makeidx}         % allows index generation
\usepackage{graphicx}        % standard LaTeX graphics tool
                             % when including figure files
\usepackage{multicol}        % used for the two-column index
\usepackage[bottom]{footmisc}% places footnotes at page bottom

% Bibliography
%\usepackage{natbib,har2nat}
\usepackage[numbers]{natbib}
\renewcommand{\refname}{Bibliografia}

\usepackage[pdfpagelabels=true, plainpages=false,
            colorlinks=true, allcolors=blue]{hyperref}

% see the list of further useful packages
% in the Reference Guide
\makeindex             % used for the subject index
                       % please use the style svind.ist with
                       % your makeindex program

%\usepackage{epigraph}

\newcommand{\fig}[4][0.50]{\begin{figure}[ht]
%\sidecaption
\includegraphics[scale=#1]{#2/#3}
\caption{#4}
\label{#3}       % Give a unique label
\end{figure}}

\newcommand{\figBot}[4][0.50]{\begin{figure}[b]
%\sidecaption
\includegraphics[scale=#1]{#2/#3}
\caption{#4}
\label{#3}       % Give a unique label
\end{figure}}

\newcommand{\figTop}[4][0.50]{\begin{figure}[t]
%\sidecaption
\includegraphics[scale=#1]{#2/#3}
\caption{#4}
\label{#3}       % Give a unique label
\end{figure}}

\newcommand{\figLatHere}[4][0.50]{\begin{figure}[ht]
\sidecaption
\includegraphics[scale=#1]{#2/#3}\caption{#4}\label{#3}
\end{figure}}

\newcommand{\figLatTop}[4][0.50]{\begin{figure}[t]
\sidecaption
\includegraphics[scale=#1]{#2/#3}\caption{#4}\label{#3}
\end{figure}}

\newcommand{\figLat}[4][0.50]{\begin{figure}[b]
\sidecaption
\includegraphics[scale=#1]{#2/#3}\caption{#4}\label{#3}
\end{figure}}

%%% Elementos adaptativos
\newcommand{\naoTerminal}[1]{\lstinline[style=antlr]{#1}}
\newcommand{\terminal}[1]{`\lstinline[style=antlr]{#1}'}
\newcommand{\uml}[1]{\lstinline{#1}}
\newcommand{\codigo}[1]{\lstinline{#1}}
\newcommand{\uba}[1]{\lstinline{#1}}
\newcommand{\instancia}[1]{\lstinline{#1}}
\newcommand{\funcao}[1]{\lstinline{#1}}
\newcommand{\estado}[1]{\lstinline{#1}}
\newcommand{\simbolo}[1]{\lstinline{#1}}
\newcommand{\regra}[1]{``\lstinline{#1}''}
\newcommand{\acao}[1]{``\lstinline{#1}''}
%%%%%%%%%%%%%%%%%%%%%%%%%%%%%%%%%%%%%%%%%%%%%%%%%%%%%%

\newcommand{\bibPasta}{../../projeto-git/bib}


\title{MTS}
\author{Eduardo Camargo}

\begin{document}

\maketitle

\tableofcontents

\begin{partbacktext}
\part{Fundamentação Teórica}
%\noindent Use the template \emph{part.tex} together with the Springer document class SVMono (monograph-type books) or SVMult (edited books) to style your part title page and, if desired, a short introductory text (maximum one page) on its verso page in the Springer layout.

\end{partbacktext}


% Este tópico é essencial na obra, e recomendamos que seja elaborado com cuidado. Seu objetivo é o de, através de uma revisão criteriosa dos conceitos científicos fundamentais em que o material do livro se baseia, recordar os assuntos tratados aos leitores que já os conhecem e ao mesmo tempo, orientar conceitualmente os leitores que não possuam conhecimentos prévios dos assuntos pertinentes a cada uma das disciplinas envolvidas nos assuntos tratados, permitindo-lhes uma compreensão, ainda que básica, do tema tratado sem que necessitem acessar outras fontes de referência, tornando assim auto-contido o material apresentado no livro.

% Nele devem ser apresentados, com relação tanto às áreas básicas como às de aplicação, todos os elementos científicos que dão suporte à tecnologia que o livro pretende sugerir. Assim, sugere-se a inclusão de uma seção de definições, na qual se esclareça:

% o significado preciso de toda a terminologia empregada,

% o significado da simbologia utilizada nas notações formais utilizadas

% no caso de assuntos inter ou multidisciplinares, a exata relação entre os elementos das diversas disciplinas envolvidas

% Na composição dos assuntos reunidos sob a classe Fundamentos, subentendem-se elementos conceituais: teorias, bases matemáticas, notações formais, modelos de representação, teoremas, demonstrações, propriedades, relações de correspondência e de equivalência.

% Não menos importantes, devem ser apresentadas, quando couber, analogias com fenômenos naturais, ouabstratos, taiscomocomocomparaçõesgeométricas, algébricas, linguísticas, físicas, biológicas, sociológicas e quaisquer outras consideradas relevantes

% A leitura do tópico Fundamentos deve permitir que o leitor exercite e aprimore seu potencial de absorver os conteúdos dos capítulos tecnológicos subsequentes, ainda que não seja um especialista em qualquer das áreas de conhecimento tratadas na obra.

	\chapter{Computação Bioinspirada}

% Ideia: sugerir que a inspiração na natureza ajuda a lidar com fenômenos ambientais.

O termo biomimética, tratado aqui como sinônimo de biomimese, biomimetismo e biônica, refere-se ao uso da natureza como inspiração para a busca de soluções para problemas humanos. Sua primeira aparição em um dicionário, \textit{Webster's Dictionary}, ocorreu em 1974 definindo-o como:  o estudo da formação, estrutura, ou função de substâncias e materiais biologicamente produzidos, bem como, o estudo de mecanismos e estruturas biológicas com o propósito de sintetizar produtos artificiais que os mimetizem. Assim, a biomimética é um campo de estudo relativamente novo que aborda o uso prático de mecanismos e funções da ciência biológica na engenharia e em outras áreas\cite{vincent06} ou, como escreve Benyus\cite{benyus03}, a biomimética é uma nova ciência que estuda os modelos da natureza e depois os imita ou se inspira neles ou em seus processos para resolver problemas humanos.

As aplicações mais comuns da biomimética relacionam-se ao desenvolvimento de produtos inspirados em estruturas biológicas. Este é o caso do trem-bala de \textit{Shinkansen} da \textit{West Japan Railway} cuja locomotiva teve o desenho inspirado na forma do bico de uma espécie de martim-pescador. Como resultado desta abordagem biomimética, houve redução substancial do nível de ruído percebido pelos passageiros, economia de 15\% de energia e alcance de velocidades 10\% superiores~\cite{biomimicry01}.

No entanto, comportamentos biológicos também podem servir de inspiração para a solução de problemas humanos complexos. Aqui, os campos da ciência da computação, da robótica e da ciência cognitiva fornecem vários exemplos: computação natural\cite{decastro06}, biorrobótica\cite{webb01} e inteligência artificial\cite{mccarthy55}, entre outros, são termos interconectados que se referem a abordagens biomiméticas na medida em que se inspiram em processos desenvolvidos pela natureza para alcançar seus objetivos.

% Ideia: apresentar a noção de computacão natural

Particularmente, a Computação Natural é formada por três ramos: (1) \textit{computação com materiais naturais}, cujo objetivo é a pesquisa de novos materiais biológicos que possam substituir a computação via silício; (2) \textit{simulação e emulação da natureza através da computação}, que é um processo sintético de se criar padrões, formas, comportamentos e organismos que podem mimetizar fenômenos naturais aumentando o entendimento da natureza e produzindo \textit{insights} sobre modelos computacionais e (3) \textit{computação inspirada pela natureza} tendo por objetivo o uso da natureza como inspiração para o desenvolvimento de técnicas de solução de problemas\cite{decastro06}. A diferença entre o segundo e o terceiro ramo é de intenção. Embora ambos possam partir de um mesmo fenômeno biológico, o segundo tem a intenção de estudo enquanto o terceiro tem a intenção de resolução de um problema. Sugere-se que ambos os ramos tenham a sua origem no trabalho de Turing. Em \cite{turing36} ele usa a mente humana como inspiração para resolver o \textit{Entscheidungsproblem} (terceiro ramo da Computação Natural). Em \cite{turing52} seu objetivo é apresentar um teste para verificar se a mente pode ser simulada (segundo ramo da Computação Natural).

% Ideia: mostrar que a origem dos modelos computacionais teve inspiração biológica
% OBS: Não valeria aqui destacar que a mente é fator biológico se descartarmos o dualismo cartesiano e adotarmos uma visão monista?

Deixando de lado o dualismo cartesiano entre mente e matéria e adotando uma posição monista na qual a mente reflete os processos biológicos subjacentes, a computação pode ser considerada, em parte, como um exemplo de tecnologia bioinspirada. Quando Turing propôs a sua máquina em \cite{turing36}, ele buscava retratar o comportamento de um computador humano\footnote{Na década de 1930, antes de existirem computadores automáticos, existiam pessoas contratadas para realizar cálculos (cômputos). O nome do cargo ou função administrativa que essas pessoas exerciam recebia o nome de \emph{computador} \cite{}.} durante a realização de cálculos. Além disso, a lógica Booleana, um elemento essencial no modelo de computador de von~Neumman, foi delineada no livro intitulado \textit{An Investigation of the Laws of Thought} \textit{cf.}~\cite{boole}. Isto sugere que Boole também trabalhou de maneira bioinspirada tendo a mente como base. Outrossim, von~Neumman parece seguir nessa linha de \emph{inspiração} biológica ao propor uma unidade responsável pelos cálculos e uma memória~\cite{Newmann:1945:FDR:1102046}. Percebe-se que Turing, Boole e von~Neumman usaram a mente humana como \emph{inspiração} para desenvolverem seus modelos computacionais. Neste sentido, é possível afirmar-se que seus modelos computacionais foram \emph{inspirados} no comportamento da mente humana: na sua origem, a computação foi bioinspirada.

% Ideia: enfatizar que a elaboração dos modelos computacionais tem foco na solução de problemas.

Convém mais uma vez distinguir a computação como simulação (segundo ramo da Computação Natural) da computação como \textit{problem-solving} (terceiro ramo da Computação Natural). A Ciência Cognitiva, por exemplo, utiliza simulações computacionais com o objetivo de estudar a mente. O próprio Turing fez uso da simulação para estudar sistemas biológicos~\cite{turing52}. Por outro lado, a Ciência da Computação estuda técnicas de \textit{problem-solving}, tais como redes neurais artificiais; computação evolucionária; inteligência coletiva; sistemas imunológicos artificiais; \textit{etc}. É neste contexto que o trabalho aqui apresentado se insere --- no âmbito da ciência da computação. Padrões biológicos que possam ser aplicados à solução de problemas humanos é um dos principais resultados alcançados. 

% Ideia: mostrar que não é suficiente a bioinspiração no caso de desenvolvimento tecnológico

Convém ainda considerar mais um ponto. Entre os algoritmos considerados ``clássicos'', muitos deles também são bioinspirados caso se considere o comportamento humano meramente como comportamento biológico. Por exemplo, os algoritmos de ordenação como \textit{bubble-sort}, \textit{selection-sort}, \textit{insertion-sort} baseiam-se em estratégias utilizadas por pessoas para ordenarem uma sequência de elementos. No entanto, alguns algoritmos como \textit{merge-sort} e \textit{quick-sort} são mecanismos criados sem inspiração biológica. Isso sugere que embora um comportamento bioinspirado possa ser útil na solução de um problema não implica que seja sempre a melhor ou a única solução. 

% Ideia: apresentar a noção de computação bioinspirada

Desta forma, em síntese, entende-se aqui Computação Bioinspirada (CB) como um particular funcionamento mecânico que visa retratar um comportamento de interesse apresentado por sistemas biológicos. A CB é um ramo da Computação Natural. Neste sentido a CB refere-se ao terceiro de seus sub-ramos apresentados.  Isso significa que a CB, objeto deste estudo, não tem como objetivo a simulação exata de todos os aspectos orgânicos de um dado fenômeno, o que se busca é a representação em termos de programa dos aspectos essenciais do fenômeno biológico, cuja utilidade é o suporte a aplicações em engenharia, finanças e outras áreas, essencialmente, distintas da biologia.

Neste ponto, cabe ressaltar o caráter heurístico da Computação Natural e, consequentemente da CB: \begin{quote}\textit{A maioria das abordagens computacionais com as quais a computação natural lida são baseadas em versões muito simplificadas de mecanismos e processos presentes nos fenômenos naturais correspondentes. São várias as razões para tais simplificações e abstrações. Primeiro de tudo, muitas simplificações são necessárias para se realizar, de forma tratável, a computação com um grande número de entidades. Também, pode ser vantajoso destacar as características mínimas necessárias para que certos aspectos particulares de um sistema sejam reproduzidos e para que propriedades emergentes sejam observadas. O nível mais apropriado para a investigação e abstração depende da questão científica proposta, do tipo de problema que se deseja resolver e do fenômeno biológico que se deseje sintetizar. A computação natural normalmente integra biologia experimental e teórica, física e química, observações empíricas da natureza e muitas outras ciências (...) para atingir seus objetivos.}~\cite{decastro06}\end{quote}

%a programação é artificial

Desta forma, a Computação Bioinspirada não procura igualar biologia à computação. Entende, ao contrário, que são dois campos distintos. Os fenômenos biológicos evoluíram naturalmente e apresentam comportamentos que, em princípio, nada têm a ver com funções computacionais. Não obstante, o olhar computacional do programador pode abstrair artificialmente aspectos dos modelos biológicos que induzem a uma interpretação computacional. Assim, a Computação Bioinspirada é uma abordagem heurística que busca estudar fenômenos naturais específicos, extraindo apenas seus elementos essenciais e passíveis de tratamento através abstrações computacionais.   

O próximo capítulo apresenta o Método de Transposição Semiótica (MTS). Trata-se, justamente, de um conjunto de passos heurísticos capazes de auxiliar o programador da Computação Bioinspirada a abstrair os elementos essenciais e computáveis de um dado fenômeno biológico.

%\section{Padrões de Biomiméticos} mais resultado do trabalho. (Já mencionado no 6º parágrafo acima)


%|=====|===========|===============================================|
%Um terceiro ponto que sugere a natureza bioinspirada da computação é a correspondência de Curry-Howard \cite{}. Provou-se nessa correspondência que uma prova escrita segundo o paradigma da matemática intuicionista equivale à um programa escrito segundo as bases do paradigma funcional (cáculo-$\lambda$). A matemática intuicionista, por sua vez, se baseia no trabalho de Brouwer \cite{brouwer} sobre o pensamento matemático.

	\chapter{Método de Transposição Semiótica}

O Método de Transposição semiótica (MTS) foi concebido como ferramenta para auxiliar a programação de computadores através de uma abordagem biomimética (Computação Bioinspirada). O programador que se utiliza de tal método busca inspiração nos comportamentos de estruturas biológicas para identificar possíveis caminhos de solução para problemas computacionais. Antes que sejam especificados os passos que integram o MTS, cabe delimitar o escopo de sua aplicação.

Uma primeira questão diz respeito à possibilidade de reconhecimento de propriedades isomórficas entre comportamentos biológicos e computacionais. Aqui, isomorfismo é entendido como definido por Hofstadter~\cite{hofstadter79}, ou seja, como a propriedade que permite duas estruturas complexas serem mapeadas entre si, de tal maneira que, para cada parte de uma dessas estruturas, exista uma parte correspondente na outra estrutura, com ambas desempenhando papéis similares. Devido à maior complexidade do campo biológico, fica clara a impossibilidade de isomorfismo completo. No entanto, o estabelecimento de alguns parâmetros restritivos pode garantir, ao menos, uma correspondência parcial capaz de mapear os elementos essenciais do fenômeno biológico em funções computacionais. Ou seja, somente os comportamentos biológicos passíveis de serem traduzidos (ou interpretados) em funções computáveis devem ser considerados. A hipótese principal do MTS é de que a semiótica aplicada pode auxiliar o programador em suas escolhas heurísticas.

Semiótica pode ser entendida como a ciência dos signos e dos processos significativos (semioses) na natureza e na cultura. A semiótica parte de um paradigma segundo o qual os atos interpretativos são compostos por encadeamento de signos. A semiótica de Peirce~\cite{peirce08}, de caráter geral, apresenta o signo como uma tríade composta por objeto, signo e interpretante. O objeto consiste em uma substância concreta do mundo fenomênico e divide-se em objeto dinâmico (o objeto em sua realidade mesma) e objeto imediato (suas qualidades perceptivas). O objeto é percebido, por certa entidade capaz de apreender suas qualidades parciais através de um signo, ou representação (ou \textit{representamen}). O efeito do signo nesta entidade produz uma interpretação sobre o objeto e sua representação, este é o interpretante. Assim, um signo peirceano é composto por um objeto, sua representação e seu interpretante. A esta inter-relação ou processo é dado o nome de semiose~\cite{peirce1940philosophical}.

Uma cadeia semiótica se instala quando o interpretante de uma semiose original torna-se o objeto de um novo signo. Isto pode se repetir indefinidamente e cada semiose aumenta a abstração ao afastar o interpretante do objeto real presente no mundo fenomênico. No entanto, um mesmo objeto real (objeto dinâmico) pode dar início a cadeias semióticas distintas já que os vários aspectos de sua realidade (objeto imediato) geram diferentes representações e interpretações de acordo com as disposições perceptivas e cognitivas da entidade afetada pelo signo. Ressalta-se que essas diferenças interpretativas podem se manifestar em qualquer ponto da cadeia semiótica, formando assim nuanças de signo.

Há de se entender que, na aplicação do MTS, vários níveis de processos semióticos estão em jogo. Primeiro, existem as semioses reais (objeto real) inerentes a cada estrutura biológica, cujas interpretações são vivenciadas apenas por elas mesmas, não sendo possível ao observador acessá-las diretamente. No entanto, o observador, através de um processo semiótico próprio, é capaz de interpretar o fenômeno (por efeito do objeto imediato) e estabelecer uma cadeia semiótica virtual, um modelo abstrato do fenômeno. Finalmente, o observador utiliza a cadeia semiótica virtual como objeto de uma nova semiose para estabelecer a correspondência computacional. Em resumo, os seguintes níveis semióticos estão presentes: (1) a cadeia semiótica inerente ao fenômeno e não acessível diretamente ao observador. O agente interpretador desta cadeia semiótica é a própria estrutura biológica afetada por ela; (2) a cadeia semiótica que se instala no observador/programador em seu ato interpretativo. O agente interpretador é o próprio observador; (3) a cadeia semiótica virtual como modelo abstrato. O agente interpretador desta cadeia semiótica seria uma instância hipotética da estrutura biológica; e (4) a cadeia semiótica responsável pela interpretação da cadeia semiótica virtual em computação. Aqui, o agente interpretador é o observador/programador. O MTS atua diretamente nas semioses do nível 2, acima, auxiliando o estabelecimento dos níveis 1 e 4.

Para estudar a computação por meio do paradigma semiótico é necessário expressá-la por meio de signos.

%Desde aqui
%Pensar junto com o Ítalo e Francisco. não seria objeto = fita na entrada, signo = estado Q correspondente e interpretante = fita na saída?

Entende-se computação como cálculo efetivo e um modelo que o realiza é a Máquina de Turing. Grosso modo, a Máquina de Turing funciona por meio de um conjunto de transições parciais que descrevem um cálculo, isto recebe o nome de \emph{especificação}. A realização da especificação faz com que a Máquina de Turing assuma determinado \emph{comportamento}, o qual, ao modificar a fita traduz-se em \emph{funcionamento}. Portanto, \emph{funcionamento} é a percepção do operador ao observar mudanças na fita; \emph{comportamento} são as ações de movimentação, leitura e escrita efetuadas pela máquina ao realizar a \emph{especificação} \cite{turing36}.

Neste sentido, a Máquina de Turing pode ser expressa em termos semióticos pela tríade especificação/comportamento/funcionamento. Tal qual ocorre com um signo peirceano a especificação é algo concreto, estável e identificável. Por outro lado o comportamento que realiza a especificação, tal qual a representação de um objeto, é variável. O funcionamento, por sua vez, produz na mente do operador um interpretante o qual será utilizado como objeto para a realização de semioses sobre o programa. Daí se afirmar que o signo computacional (ou signo de Turing) ser composto por especificação, comportamento e funcionamento. Cabe lembrar que o signo é necessariamente uma construção mental e portanto este é o signo que se costuma formar na mente ao compreender a Máquina de Turing.

Aplicando recursivamente a análise semiótica sobre o comportamento tem-se que a Máquina~de~Turing é um instrumento para o cálculo de números computáveis, onde, para Turing um número é computável se puder ser escrito por uma máquina em termos finitos \cite{turing36} ou por uma sobreposição de intervalos \cite{turing36_errata}. Uma vez que se está propondo usar a semiótica na computação convém explorar a ideia de signo computável. Em termos peirceanos um número pode ser classificado como legi-signo \cite{winfried}, assim, um legi-signo é computável se o \textit{representamen} puder ser escrito por uma Máquina de Turing em termos finitos.

Existem seis classes de legi-signos \textit{cf.}~\cite{teoriaSignos_Santaella}: 1)  o icônico-remático; 2) indicativo-remático; 3) indicativo-dicente; 4) simbólico-remático; 5) simbólico-dicente; e 6) simbólico-argumental. Um legi-signo é icônico quando o interpretante representa uma similaridade; indicativo quando o interpretante representa um ponteiro; e simbólico quando o interpretante representa hábitos de uso. Um legi-signo é remático quando o interpretante representa uma possibilidade; dicente quando o interpretante representa um fato; e argumento quando o interpretante representa uma lei.

Em computação o legi-signo assume a forma de um \emph{tipo abstrato} onde o objeto é uma referência à um objeto no domínio do problema, o \textit{representamen} um simbolo computável e o interpretante é definido \textit{a priori} pelas operações do tipo (ou funções de transição da Máquina de Turing). Isto, por sua vez propicia as quasi-semioses \cite{winfried}, isto é, semioses definidas \textit{a priori} e de cunho automático; um paralelo desta idéia na computação são comportamentos polimórficos ou adaptativos.

% até aqui

Assim, existem dois níveis de interpretação semiótica na Máquina de Turing. O primeiro nível consiste na formação do signo na mente do \emph{operador} e o segundo nível a forma pela qual os signos foram desenvolvidos e realizados na máquina pelo \emph{programador}.

Considerando que é possível expressar a Máquina de Turing em termos semióticos e dado que o mesmo é possível para organismos vivos e meio-ambiente como a zoossemiótica, biossemiótica e os processos microbiológicos, genéticos e evolutivos~\cite{noth95, noth96}. É possível sugerir que o campo da semiótica possa ser utilizado como campo intermediário na transposição dos aspectos essenciais de um determinado fenômeno biológico do campos natural ao campo computacional, agindo assim como elemento heurístico.

Partindo desta concepção, propõe-se neste estudo uma heurística de transposição semiótica intitulada Método de Transposição Semiótica (MTS). O MTS fundamenta-se em três corpos teóricos, dois deles de base semiótica, a teoria da significação de Uexküll~\cite{uexkull34, uexkull82} e o estruturalismo hierárquico de Salthe~\cite{salthe85}; e o outro, baseado na inteligência artificial, mais precisamente, na arquitetura de subordinação de Brooks~\cite{brooks99a, brooks99b, brooks99c}.

\section{Arquitetura de subordinação de Brooks}

Brooks desenvolveu na década de 1980 uma crítica consistente ao modelo de IA clássica, marcadamente baseada no representacionalismo simbólico e sua estratégia \textit{top-down}. Para ele, a IA clássica, ao partir do pressuposto de que cognição e representação seriam a mesma coisa, passa a concentrar esforços na simulação de atividades cognitivas superiores para, então, tentar simular atividades mais básicas. Em termos peirceanos, Brooks considera que a terceiridade, enquanto representação de mundo, não deva ser dada \textit{a priori} mas alcançada pelo processo normal de construção do signo. Ou seja, que o automato, por meio de sensores, deve se utilizar das possibilidades de encontros com os objetos do mundo (primeiridade) para ajustar seu conjunto de regras (terceiridade), e isto acarreta um comportamento (secundidade) adequado ao contexto específico no qual ele está inserido. Portanto, o que Brooks sugere é que a programação deva ter foco na secundidade e não na terceiridade.

Na terminologia de \textit{problem-frames} \textit{cf.}~\cite{Jackson:2000:PFA:513720} significa que não deve haver um modelo \textit{a priori} que represente o mundo do problema e que a máquina deve apresentar um comportamento polimórfico-adaptativo capaz de se adequar ao mundo do problema. Ao construir seus robôs, Brooks e sua equipe chegam à conclusão de que quando são examinados níveis muito simples de inteligência, percebe-se que as representações explícitas e modelos de mundo apenas atrapalham, e levantam a hipótese de que a representação do mundo do problema seria uma unidade errada de abstração ao construir a maior parte dos \emph{sistemas inteligentes}~\cite{brooks99a}.

Para Brooks, a história da evolução biológica desde as células simples até a inteligência humana sugere que as capacidades superiores humanas como linguagem, razão e capacidade de solução de problemas complexos só foram possíveis após a evolução ter proporcionado soluções reativas para problemas menores como a capacidade de organismos primitivos se moverem num ambiente dinâmico, sentindo este ambiente e agindo no sentido da autopreservação e reprodução. Isto justifica o  estudo da inteligência segundo uma estratégia \textit{bottom-up} focada em sistemas físicos (robôs móveis, por exemplo) situados num mundo e autonomamente cumprindo tarefas variadas. Alguns desses trabalhos são baseados em princípios da engenharia e outros são firmemente baseados em inspirações biológicas~\cite{brooks99a, brooks99b, brooks99c}. Em suma, Brooks argumenta que os projetos baseados em circuitos são tudo o que é necessário à IA, que a representação e o raciocínio são incômodos, dispendiosos e desnecessários.

Este é o aspecto conceitual da Arquitetura de Subordinação~\cite{brooks99a, brooks99b, brooks99c}. Ainda para Brooks, esta concepção é realizada computacionalmente por meio do encadeamento de camadas compostas por máquinas de estados finitos. Ou seja, ele sustenta que todo trabalho de engenharia tem a necessidade de decompor um sistema complexo em partes menores, ou módulos, para construir cada parte e então integrá-las num sistema completo. Para proceder a esta modularização, existem duas estratégias básicas: modularização por função ou modularização por atividade. Na construção de seus robôs, Brooks opta pela segunda. Com esta estratégia Brooks consegue conectar cada \textit{input} diretamente a uma saída gerando camadas comportamentais \textit{cf.}~\cite{parnas}.

A vantagem deste método é que proporciona uma construção incremental, das atividades mais simples às mais complexas. Por exemplo, se for necessário fazer uma alteração no módulo \textit{andar}, não ocorrem mudanças indesejadas em outros módulos. Às vezes esta abordagem é chamada de comportamentalista e suas premissas são: \textbf{objetos situados}, os robôs são situados num mundo, não lidando com descrições abstratas mas com o aqui e agora deste mundo; \textbf{objetos corporificados}, os robôs possuem um corpo e experienciam o mundo através de seus sensores e atuadores; \textbf{inteligência}, os robôs são percebidos como inteligentes, mas isto não deriva simplesmente de sua capacidade computacional mas também do acoplamento com o mundo; e \textbf{emergência}, a inteligência do sistema emerge da interação do sistema com o mundo e às vezes de interações indiretas de seus componentes. Portanto, a nova IA de Brooks baseia-se na hipótese da fundamentação física que estabelece que para se construir um sistema inteligente é necessário ter suas representações  fundamentadas no mundo físico.

\section{Estruturalismo Hierárquico de Salthe}

O estruturalismo hierárquico, tal qual é abordado na obra de Salthe~\cite{salthe85}, é um estudo de meta-teoria científica a partir da perspectiva da biologia e pretende lidar com a representação das coisas do mundo e suas relações. É uma teoria sistêmica que, apesar de originar-se nas ciências biológicas, sua aplicação pode ser útil em qualquer outro campo do conhecimento.

Para El-Hani e Queiroz~\cite{queiroz07}, o trabalho de Salthe é uma proposta de abordagem de sistemas complexos que apresenta o sistema triádico básico de influência peirceana como elemento fundamental. Segundo o estruturalismo hierárquico, para se descreverem as interações e processos entre entidades fundamentais num dado sistema precisa-se: primeiro, identificar em  qual nível focal o processo se instala; segundo, investigar tal processo em termos das entidades no nível anterior; e, terceiro, considerar as entidades que operam num nível superior. Para processos semióticos, consideram-se os níveis como: focal, micro-semiótico e macro-semiótico. O nível micro-semiótico apresenta todas as tríades possíveis a partir de um mesmo objeto dinâmico, assim, certa semiose que surge no nível focal seria a efetivação do conjunto de algumas possibilidades armazenadas no nível micro-semiótico; o nível macro-semiótico, por sua vez, representa o conjunto de semioses possíveis a um determinado sistema funcionando como elemento limitador.

Complementando, o nível inferior, micro-semiótico ou micro-sistêmico é o único do qual emana a capacidades criativa do sistema. As possibilidades que devem ocorrer no nível focal são geradas a partir deste nível e só algumas delas serão realizadas quando confrontadas com as restrições do nível superior~\cite{salthe85}.

No estruturalismo hierárquico, as noções de \textbf{objeto}, \textbf{entidade} e \textbf{coisa} estão interligadas. Assim, objetos são experienciados implicando a presença de um sujeito, as coisas são acreditadas e as entidades definidas. Para que uma coisa possa ser considerada uma entidade, deve apresentar uma \textbf{fronteira} discernível, deve ser um sistema cibernético bem \textbf{integrado} e, finalmente, deve apresentar \textbf{continuidade espaço-temporal}. Assim, uma entidade se define como algo de certo tamanho, distinto de sua cercania, possuindo sub-partes em relações cibernéticas e estabilidade ao longo do tempo~\cite{salthe85}.

Quando um determinado fenômeno é analisado sob a ótica do estruturalismo hierárquico, deve-se proceder à escolha do nível focal segundo as necessidades e interesses do observador. Um quadro completo de níveis hierárquicos pode ser descrito através dos seguintes níveis: atômico, molecular, organelas intracelulares, células, tecidos, órgãos, aparelhos, organismo, nicho ecológico, ecossistema, meio-ambiente, de acordo com necessidades específicas alguns desses níveis podem ser condensados e outros suprimidos, por exemplo, uma estrutura possível seria considerar como nível focal o nível das células e suas organelas. Neste caso, pode-se entender o nível molecular como o nível inferior e o nível do organismo como nível superior, suprimindo os níveis dos tecido, órgãos e aparelhos~\cite{salthe85}.

\section{A teoria da significação de Uexküll}

A teoria biossemiótica de Jakob von Uexküll, ou teoria da significação biológica, apresenta-se como uma possibilidade de grande alcance para o estudo das relações entre animais e o meio-ambiente. Esta abordagem, segundo Thure von Uexküll~\cite{thure04}, adota o ponto de vista sistêmico, negando tanto o objetivismo positivista quanto o subjetivismo idealista e aponta para a interação entre sujeito e objeto como  uma inter-relação em um todo maior.  O  objeto não pode ser estudado separadamente de seu meio-ambiente. Uexküll cria, então, o conceito de \textit{Umwelt} que seria o segmento ambiental de um organismo, definido por suas capacidades específicas da espécie, tanto receptoras quanto efetoras. Segundo Uexküll:

\begin{quote}\textit{Todos os animais, do mais simples ao mais complexo, ajustam-se dentro de seus mundos únicos com igual completude. Um mundo simples corresponde a um animal simples, um mundo bem-articulado a um animal complexo.}~\cite{uexkull34} \end{quote}

Para Uexküll~\cite{uexkull34, uexkull82}, as relações entre sujeito e objeto que definem o \textit{Umwelt} de uma determinada espécie são representadas através de um diagrama que ele denomina círculo funcional e que caracteriza a dinâmica bem ajustada entre o objeto (\textit{meaning-carrier} ou portador da significação) e o sujeito (\textit{meaning-receiver} ou receptor da significação). O objeto é portador de dois tipos de pistas referentes ao processo de significação que está em jogo, \textbf{pista perceptiva} e \textbf{pista operacional}. Através da primeira, o objeto é percebido pelo  animal ao captar em seu órgão receptor um determinado \textbf{sinal perceptivo}. Segundo as características da sua espécie, o órgão receptor inicia a transmissão de impulsos correspondentes em direção ao \textbf{órgão operacional}. O próprio objeto fornece a \textbf{pista operacional} através do \textbf{sinal operacional} que extingue a pista perceptiva completando o círculo funcional. Os sinais provenientes do objeto são inerentes ao objeto e as pistas que projeta são inerentes às capacidades de percepção do animal, ou seja, sinais são traduzidos em pistas, segundo a característica de cada espécie. Ainda, em outras palavras, a interação com um objeto causa no animal a formação de um signo. Como exemplo: o ácido butírico exalado pelas glândulas dos mamíferos (sinal) seria captado pelo órgão receptor  de outro animal qualquer e percebido como estímulo olfativo (pista) (Fig~\ref{0203circulofuncional-eps-converted-to}).

\figLatTop[0.8]{\eduPastaFig}{0203circulofuncional-eps-converted-to}{Circulo funcional de Uexküll.}

No modelo de Uexküll, estruturas mais simples como células também apresentam características de \textit{Umwelt}. O \textit{Myxomyces}, um tipo de fungo, apresenta seus esporos em estado inicial de desenvolvimento como células parecidas com amebas com movimentos livres, alimentando-se de bactéria floral. Nas palavras de Uexküll:

\begin{quote}\textit{Somos forçados a atribuir \textit{Umwelt}, mesmo que limitado, às células-fúngicas livres e vivas, um \textit{Umwelt} comum a cada uma delas, no qual a bactéria  contrasta com o entorno, como portadoras de significação. como alimento e, assim, são percebidas causando reação. Por outro lado, o fungo, composto de muitas células unitárias, é uma planta que não apresenta \textit{Umwelt}t animal - é apenas cercado por um tegumento-habitável constituído de fatores de significação}~\cite{uexkull34} \end{quote}

Os vegetais, segundo Thure von Uexküll~\cite{thure04}, como apenas tegumentos habitáveis, não detectam qualidades especiais nos sinais perceptivos e operacionais, portanto tais sinais não geram pistas e se mostram suficientes para a realização de processos sígnicos.  Neste caso, não seria o círculo funcional a descrever as semioses, mas sim o sistema retroativo \textit{feedback system}. Os signos (sinais) perceptivos são codificados por um receptor e os signos operacionais, através da atividade de um efetor, ajustam o valor real de um sistema variável (o tegumento habitável de um vegetal ou célula), fazendo este valor concordar com o valor referencial requerido.

Desta forma, os processos semióticos distribuem-se segundo níveis hierárquicos. Para Thure von Uexkül~\cite{thure92}, essa hierarquia pode ser expressa em termos do usuário da significação (\textit{meaning-utilizer}): sistemas de signos intracelulares, no qual as organelas seriam  \textit{meaning-utilizers}; sistemas de signos intercelulares e celulares, que corresponderia à endossemiótica; sistemas de relações sígnicas entre animais ou zoossemiótica e sistemas sígnicos entre grupos sociais através da linguagem humana. Para ele, o sistema de Jakob von Uexküll estaria confinado ao segundo e terceiro desses tipos de sistemas. Para Sebeok \cite{sebeok99}, no entanto, a endossemiótica também teria em seu campo de estudo as relações sígnicas entre organelas celulares, células, tecidos, órgãos, além de relações genéticas. Assim, a teoria da significação de Uexküll, segundo Ziemke~\cite{ziemke01}, pode ser de grande ajuda no entendimento do uso de signos e representações pelos organismos vivos e na compreensão das possibilidades e limitações de autonomia e semiose em organismos artificiais~\cite{ziemke01}.

\section{Premissas do MTS}

O estudo de um determinado fenômeno biológico origina, através do MTS, um meta-modelo computacional que representa as semioses essenciais subjacentes a tal fenômeno. Para a construção deste meta-modelo, o MTS utiliza a arquitetura de subordinação de Brooks para determinar as camadas responsáveis por comportamentos individuais. A composição desses comportamentos individuais dá origem ao comportamento global do sistema sendo que camadas superiores prevalecem sobre camadas inferiores (Fig~\ref{0205subordinacao-eps-converted-to}). Como critério de decomposição das camadas foi utilizada a decomposição por responsabilidades proposto por Parnas em \cite{parnas72}. Desta forma, tendo em vista a implementação computacional final, são identificados com base no animal os possíveis módulos biológicos que podem ser implementados em módulos computacionais.

\figLatHere[0.6]{\eduPastaFig}{0205subordinacao-eps-converted-to}{Arquitetura de subordinação baseada no trabalho de Brooks.}

O tratamento da complexidade inerente aos sistemas biológicos requer um mecanismo capaz de auxiliar a pesquisa no encontro das entidades biológicas relevantes à abstração de um algoritmo biológico. Para tanto, o MTS utiliza o estruturalismo hierárquico de Salthe. Para Salthe, deve-se proceder à escolha do nível focal segundo as necessidades e interesses do observador, consequentemente, os níveis inferior e superior se estabelecem. O nível focal é onde se realiza a cadeia semiótica correspondente a um determinado fenômeno; o inferior é o nível das possibilidades ou nível iniciador onde se encontram as semioses potenciais; e o nível superior é o nível das restrições segundo as características de uma determinada espécie~\cite{salthe85} (Fig~\ref{0206estruturalismohierarquico-eps-converted-to}).

\figLatHere[0.6]{\eduPastaFig}{0206estruturalismohierarquico-eps-converted-to}{Estrutura hierárquica esquemática.}

A arquitetura de subordinação e o estruturalismo hierárquico atuam como elementos organizadores, sendo que o último, através do conceito de nível focal, também auxilia no direcionamento do olhar do pesquisador em busca das entidades relevantes ao algoritmo biológico em questão. Mas, quais seriam essas entidades?  Adota-se, aqui que uma entidade relevante é aquela responsável por uma semiose. Mas, existem semioses no contexto biológico? Para responder a esta questão a pesquisa adota a teoria da significação de Uexküll como corpo teórico que sustenta a resposta positiva. Adota também o pressuposto de Sebeok~\cite{sebeok99} de que processos biológicos podem ser vistos como processos triádicos, mais adequado ao estudo de semioses em nível celular, ultrapassando, assim, o conceito de círculo funcional de Uexküll.

Finalmente, como representação gráfica de uma semiose triádica adota-se o \textit{tripod} que, segundo Queiroz~\cite{queiroz04}, é a estrutura que melhor representa esta relação, pois o triângulo é, na verdade uma relação de pares de termos sendo que nenhuma combinação de vértices produz uma relação triádica (figura~\ref{0207tripod-eps-converted-to})

\figLatHere[0.6]{\eduPastaFig}{0207tripod-eps-converted-to}{O \textit{tripod} como representação gráfica de uma semiose.}

Essas passagens são descritas graficamente na Fig~\ref{0208conjunto-eps-converted-to}. Onde, partindo-se de um determinado fenômeno biológico (A), são identificadas, no nível focal e de acordo com uma arquitetura de subordinação adequada, as semioses relevantes à abstração do algoritmo biológico dos comportamentos em estudo, esses elementos formam um modelo semiótico que representa o algoritmo em questão (B); cada semiose representada no modelo semiótico é transposta como uma entidade computacional gerando um meta-modelo computacional (C) finalizando a transposição e concluindo o MTS. A partir do meta-modelo, várias aplicações podem ser derivadas dando origem a dispositivos computacionais específicos a diferentes domínios (D).

\figLatTop[0.6]{\eduPastaFig}{0208conjunto-eps-converted-to}{Método de Transposição Semiótica - diagrama geral.}

Com base nisso o MTS considera o \textit{tripod} como uma representação da estrutura de um \emph{tipo abstrato} e seu comportamento é determinado por meio de uma máquina de estados finitos as quais são encadeadas para formarem o construto computacional. Como o polimorfismo é um elemento central das construções biomiméticas optou-se como tecnologia os autômatos adaptativos devido a possuir poder de expressão conveniente para este tipo de sistema \cite{italo}.


\begin{partbacktext}
\part{Linguagem para Descrição de UBAs}
\end{partbacktext}


\begin{partbacktext}
\part{Modelagem Biomimética Dirigida por UBAs}
\noindent Use the template \emph{part.tex} together with the Springer document class SVMono (monograph-type books) or SVMult (edited books) to style your part title page and, if desired, a short introductory text (maximum one page) on its verso page in the Springer layout.

\end{partbacktext}

	\chapter{Habituação e sensibilização: UBA inspirada em comportamento celular (UBA-HS)}

Os circuitos neurais subjacentes à aquisição de memória e aprendizagem foram estudados por Kandel, ganhador do Prêmio Nobel em Fisiologia ou Medicina em 2000. Uma parte importante dos seus estudos tratou do mapeamento dos processos neurais do sistema nervoso de organismos simples. O animal escolhido por Kandel para suas pesquisas foi o \textit{Aplysia californica}~\cite{kandel06}.

A ideia central de Kandel era simular nas células nervosas da aplísia os padrões de estimulação sensorial que Pavlov havia empregado em seus experimentos com aprendizagem, traduzindo, assim, esses protocolos comportamentais em protocolos biológicos. Afinal de contas, a habituação, a sensibilização e o condicionamento clássico --- os três protocolos de aprendizagem descritos por Pavlov --- constituem, essencialmente, séries de instruções sobre a forma como um estímulo sensorial deve ser apresentado, sozinho ou em combinação com outro estímulo sensorial, para produzir aprendizagem. Assim, conseguiu produzir, nos caminhos neurais da aplísia, padrões de atividade similares aos que são apresentados por animais submetidos a treinamento nessas três tarefas de aprendizagem. Foi possível determinar, então, de que maneira as conexões sinápticas são modificadas pelos padrões de estímulos que simulam diferentes formas de aprendizagem~\cite{kandel06}.

Kandel optou pelo reflexo de retração da guelra como o comportamento a ser estudado. A guelra é um órgão externo utilizado pela aplísia para respirar. Esse órgão situa-se numa cavidade da parede corporal denominada cavidade do manto e fica encoberto por uma lâmina de pele que é chamada de prega no manto. A prega do manto termina no sifão, um tubo que expele água e resíduos da cavidade do manto. Basicamente, para gerar a habituação, kandel provoca um toque (evento não nocivo) no sifão, o que provoca a retração defensiva da guelra. No entanto, com a repetição deste estímulo, gradativamente a intensidade da retração da guelra diminui: o animal se habitua ao estímulo. O efeito da habituação pode durar de minutos a semanas dependendo da dinâmica das seções de treinamento, gerando um aprendizado de curto ou de longo prazo. Para a sensibilização, Kandel aplica um choque (estímulo nocivo) à cauda do aplísia. Nesta nova situação, seguindo-se um toque no sifão (evento não nocivo) a retração da guelra ocorre numa intensidade muito maior. Este efeito também pode durar de alguns minutos a semanas de acordo com a dinâmica do treinamento. Assim, diante de um único evento (toque no sifão), três diferentes respostas podem ocorrer: normal (para animais não treinados); habituada (de curto ou longo prazos) quando o animal toma o estímulo como ruído; e sensibilizada (de curto ou longo prazos) quando o animal toma o estímulo como sinal~\cite{kandel06, kandel00, kandel06a, kandel06b, kandel01, kandel70}.

\section{Modelagem semiótica da UBA-HS: passos 1 ao 5 do MTS}

\subsection{Passo 1: análise preliminar}

Os fenômenos de habituação e sensibilização são essenciais ao desenvolvimento do aplísia. A habituação permite que o comportamento do animal adquira foco. O animal imaturo quase sempre responde com exagero a estímulos não ameaçadores. Habituar-se a tais estímulos faz com que o animal se concentre em estímulos realmente importantes para a organização de sua percepção e, consequentemente, para sua sobrevivência. Por outro lado, a sensibilização é a imagem espelhada  da habituação e tem por função fazer com que o animal, após ser exposto a um estímulo verdadeiramente ameaçador, tenha respostas acentuadas a qualquer estímulo, mesmo os não nocivos. É uma espécie de medo aprendido que aumenta o nível de atenção do animal em contextos específicos~\cite{kandel06}.

A capacidade de aprendizagem implícita da aplísia através dos processos neurais de habituação e sensibilização apontam para um sistema bastante eficaz de reconhecimento de signo e ruído, o que permite ao animal transitar em seu ambiente dirigindo a atenção para eventos realmente relevantes a cada momento. Analogamente, há vários sistemas humanos nos quais a identificação de eventos relevantes, ou informação útil (signos) dentre uma infinidade de eventos aleatórios com baixa diferenciação, ou informação inútil (ruídos) é uma questão de difícil solução: prospecção de informação em grandes volumes de dados (\textit{big data})~\cite{zikopoulos12}, detecção de terremotos~\cite{silver13}, segurança de software~\cite{singer14}, entre outras, poderiam se beneficiar de um sistema eficiente que consiga aprender a reconhecer quando um determinado \textit{input} deve ser considerado ou não. Esta analogia, diferenciação entre signo e ruído como função a ser transposta, aponta para algumas possibilidades interessantes no desenvolvimento de softwares inspirados no algoritmo biológico subjacente às capacidades de aprendizagem da aplísia.

\subsection{Passo 2: definição da arquitetura de subordinação}

Levando-se em conta as percepções e ações envolvidas nos experimentos de Kandel, três camadas de subordinação são reconhecidas no fenômeno de habituação e sensibilização da aplísia. Primeiramente, existe uma camada de comportamento normal relacionada às percepções e ações básicas. Acima desta camada encontra-se a camada de habituação que modula e prevalece à primeira e, finalmente, uma terceira camada, a da sensibilização que também modula a camada de comportamento normal e prevalece a ela e também à camada de habituação (Fig~\ref{0301subordinacaoaplisia-eps-converted-to}).

\figLatHere[0.6]{\eduPastaFig}{0301subordinacaoaplisia-eps-converted-to}{Arquitetura de subordinação abstraída dos fenômenos de habituação e sensibilização do \textit{Aplysia californica}.}

\subsection{Passo 3: definição do nível focal}

Os estudos de Kandel procuram reconhecer os caminhos neurais subjacentes aos comportamentos de habituação e sensibilização da aplísia, desta forma, a própria ação de Kandel aponta para o nível focal em questão, ou seja, o nível celular e de suas organelas. Consequentemente, o nível inferior ou micro-semiótico, iniciador dos processos, é o nível molecular e o nível superior ou macro-semiótico, que apresenta as restrições naturais, é o nível do organismo~\cite{kandel06, kandel00} (Fig~\ref{0302nivelfocalaplisia-eps-converted-to}).

\figLatHere[0.6]{\eduPastaFig}{0302nivelfocalaplisia-eps-converted-to}{Representação dos níveis hierárquicos abstraídos dos fenômenos de habituação e sensibilização do \textit{Aplysia californica}.}

\subsection{Passo 4: levantamento das semioses relevantes}

Heuristicamente, são considerados aqui os seguintes neurônios como responsáveis pelas semioses que atuam nos processos de habituação e sensibilização da aplísia: neurônio sensor do sifão, neurônio motor da guelra e interneurônio facilitador. No entanto, como a maioria dos processos essenciais relativos ao fenômeno em estudo ocorrem em subpartes do neurônio sensor do sifão, para a maioria das semioses levantadas, não é considerado o neurônio global como a entidade biológica responsável, mas sim uma de suas subpartes. Mais especificamente, considera-se que apenas as semioses S3 e S8 (abaixo) ocorrem por ação global de neurônios (respectivamente, neurônio motor da guelra e interneurônio facilitador), para todas as outras, consideram-se subpartes do neurônio sensor do sifão como entidades biológicas responsáveis.

Assim, seguem abaixo as descrições das semioses relevantes encontradas no nível focal referentes aos fenômenos de habituação e sensibilização do \textit{Aplysia californica}, e estão divididas segundo a arquitetura de subordinação definida no passo 2.

\subsubsection*{Camada de comportamento normal}

Esta camada comportamental pode ser representada heuristicamente por cinco semioses relevantes: S1, S2 e S3 correspondem à cadeia semiótica básica e as semioses S4 e S5 à modulação desta cadeia.

\begin{itemize}

	\item \textbf{Semiose 1 - Mecanismo de disparo do neurônio sensor do sifão}

	Decorre do toque no sifão a possibilidade de se instalar o processo de retração da guelra. Neste instante, há um rearranjo dos íons através dos canais iônicos da zona de disparo do neurônio sensor. Se este rearranjo, de configuração condizente com a intensidade do toque, for suficiente para que o limiar do potencial de repouso seja rompido (-55 mV), instala-se o processo, caso contrário, nada ocorre.

	Em termos semióticos (tríade Objeto/Signo/Interpretante) tem-se: o rompimento do limiar como o sinal perceptivo que designa o objeto (O), o rearranjo nos íons como objeto percebido que designa o signo (S) e o potencial de ação resultante como sinal operacional que designa o interpretante (I)~\cite{kandel06, kandel00, lent01}.

	\figLatHere[0.6]{\eduPastaFig}{0303s1-eps-converted-to}{Características da semiose S1.}

	A zona de disparo é a entidade biológica responsável pela primeira semiose que dá origem à cadeia semiótica que se instala no nível focal. Ela deve: 1. Interpretar se um estímulo é suficientemente forte para gerar um potencial de ação; e 2. Calibrar o potencial de ação eventualmente gerado, de acordo com a intensidade do toque. No caso dos experimentos de Kandel, os toques ocorrem sempre com intensidades semelhantes, tornando esta primeira entidade interpretadora responsável apenas pela primeira de suas atribuições, ou seja, interpretar se houve ou não o alcance do limiar. Trata-se de uma interpretação binária.

	Na continuidade do processo em direção à semiose
	%S2\nomenclature{S1 a S10}{Semioses S1 a S10},
	vale ressaltar o caráter heurístico deste passo. Após o primeiro potencial de ação na zona de disparo do neurônio sensor do sifão, ocorrem sucessivos novos potenciais de ação ao longo do axônio até que seja atingida a zona ativa no terminal do mesmo neurônio. Esta cadeia de potenciais de ação apenas transmite as características do primeiro potencial, são apenas repetições necessárias devido às características físicas do axônio (comprimento), não se tratando de novas semiose no sentido qualitativo.

	Desta forma, tais semioses não são consideradas na composição do modelo semiótico para o fenômeno em estudo. Como destacado anteriormente, este passo também tem a função de eliminar semioses não relevantes para a abstração do algoritmo biológico.

	A Fig~\ref{0303s1-eps-converted-to} refere-se às características da semiose S1.

	\item \textbf{Semiose 2 - Mecanismo de exocitose}

		\figLatHere[0.6]{\eduPastaFig}{0304s2-eps-converted-to}{Características da semiose S2.}

	Quando o potencial de ação atinge a zona ativa do axônio do neurônio sensor do sifão, ocorre uma semiose qualitativamente diferente das anteriores. A zona ativa é rica em canais de $Ca^{++}$, dependentes de voltagem, desta forma ocorre um grande fluxo de íons para o interior da membrana, provocando o fenômeno da exocitose que é a fusão da membrana das vesículas com a face interna da membrana do terminal sináptico. Assim as moléculas neurotransmissoras (glutamato) são liberadas na fenda sináptica.

	A quantidade de neurotransmissores
	%(Qn\nomenclature{Qn}{Quantidade de neurotransmissores})
	que participa da exocitose é proporcional à frequência de PA que chega à zona ativa e às quantidades: 1. De vesículas mobilizadas
	%(Qv\nomenclature{Qv}{Quantidade de vesículas mobilizadas});
	e 2. De conexões sinápticas disponíveis
	%(Qs\nomenclature{Qs}{Quantidade de conexões sinápticas disponíveis}).
	Como a frequência de PA é proporcional à intensidade do estímulo inicial na zona de disparo e, no trabalho de Kandel, esta intensidade é sempre a mesma, a quantidade de neurotransmissores que participa da exocitose pode ser considerada determinada apenas pelos valores de Qv e Qs (Qn = Qv * Qs), influências respectivas de S4 e S5, abaixo.

	Em síntese: o PA atua como objeto
	%(O\nomenclature{O, S e I}{Respectivamente, objeto, signo e interpretante}
	para esta nova semiose cujo signo (S) é o fluxo de grande quantidade de íons $Ca^{++}$ para o interior da célula resultando na exocitose (I)~\cite{kandel06, kandel00, lent01}.

	A Fig~\ref{0304s2-eps-converted-to} refere-se às características da semiose S2.

	\item \textbf{Semiose 3 - Mecanismo motor}

	\figLatHere[0.6]{\eduPastaFig}{0305s3-eps-converted-to}{Características da semiose S3.}

	De maneira heurística, a semiose S3 é considerada aqui como resultado da atuação global do neurônio motor da guelra e não de suas subpartes. Assim, o glutamato na fenda sináptica é captado pelos receptores do neurônio motor da guelra e se apresenta como objeto (O) de uma terceira semiose. Isto provoca uma configuração orgânica específica neste neurônio (S), provocando reações físicas na região de inervação da guelra (I) resultando  em retração. A retração será tão forte quanto for a quantidade de glutamato na fenda sináptica~\cite{kandel06,kandel00}.

	A Fig~\ref{0305s3-eps-converted-to} refere-se às características da semiose S3.

	\item \textbf{Semiose 4 - Mecanismo de endocitose}

	A exocitose esvazia as vesículas de glutamato na zona ativa do neurônio sensor do sifão. Então, um mecanismo oposto, a endocitose, instala-se na mesma região restaurando tais vesículas. Assim, o esvaziamento das vesículas (O) provoca uma configuração orgânica específica na zona ativa (S) cujo resultado é a recomposição das vesículas (endocitose) (I).

		\figLatHere[0.6]{\eduPastaFig}{0306s4-eps-converted-to}{Características da semiose S4.}

	A quantidade de vesículas (Qv) que estará disponível para a próxima exocitose é modulada pelo histórico dos eventos mediadores e moduladores que levam à habituação e sensibilização de curto prazo, ou é modulada pela ausência desses eventos, mantendo ou ajustando Qv para valores compatíveis com o comportamento normal~\cite{kandel06,kandel00}.

	A semiose S4 atua como moduladora da semiose S2 restabelecendo as vesículas de glutamato que estarão disponíveis quando houver uma nova exocitose. As condições de habituação e sensibilização determinam se a próxima reação do sistema frente a um novo toque no sifão ocorrerá em intensidade mais alta ou mais baixa.

	A Fig~\ref{0306s4-eps-converted-to} refere-se às características da semiose S4.

	\item \textbf{Semiose 5 - Mecanismo construtor}

	Este mecanismo regula a quantidade de conexões interneurônios (Qs) e contribui para a instalação da memória de longo prazo tanto para a habituação quanto para a sensibilização, e ocorre através de processos localizados em todo o neurônio sensor do sifão. Seu objeto (O) é a presença de certa concentração proteica específica que conduz a novas configurações orgânicas (S). O aumento ou diminuição do número de conexões é o interpretante (I) desta semiose.

			\figLatHere[0.6]{\eduPastaFig}{0307s5-eps-converted-to}{Características da semiose S5.}

	A quantidade de conexões sinápticas (Qs) que estará disponível para a próxima exocitose é modulada pelo histórico dos eventos mediadores e moduladores que levam à habituação e sensibilização de longo prazo; ou é modulada pela ausência desses eventos, mantendo ou ajustando Qs para valores compatíveis com o comportamento normal~\cite{kandel06,kandel00}.

	A semiose S5 também atua como moduladora da semiose S2 aumentando ou diminuindo a quantidade de conexões entre os neurônios sensor do sifão e motor da cauda. O histórico de eventos, aqui, determina o tempo de persistência dos comportamentos de habituação de sensibilização.

	A Fig~\ref{0307s5-eps-converted-to} refere-se às características da semiose S5.

\end{itemize}

\subsubsection*{Camada de habituação}

Esta camada comportamental pode ser representada heuristicamente por duas semioses relevantes:

\begin{itemize}
	\item \textbf{Semiose 6 - Mecanismo de habituação de curto prazo}

	\figLatHere[0.6]{\eduPastaFig}{0308s6-eps-converted-to}{Características da semiose S6.}

	Quando um novo estímulo não prejudicial é aplicado ao sifão da aplísia, o animal não treinado responde com a retração da guelra num reflexo de proteção ou atenção. Este novo estímulo é tomado pelo animal como algo potencialmente prejudicial. Porém, com a repetição de tal estímulo, a aplísia tende a diminuir sua reação até ignorá-lo. Isto se dá devido ao enfraquecimento da efetividade da transmissão sináptica por parte do neurônio sensor, através da diminuição da quantidade de neurotransmissores liberados nos terminais pré-sinápticos do neurônio sensor. O mecanismo responsável por essa diminuição na liberação de neurotransmissores ainda não é bem conhecido, acredita-se, porém, que tal fato seja devido à redução da mobilização de vesículas na zona ativa~\cite{kandel06, kandel00}.

	Para se gerar uma habituação de curto prazo, que persiste por alguns minutos, aplicam-se ao animal 10 estímulos numa única sessão. Após esta sessão, percebe-se que a intensidade da resposta é um vigésimo do original.  Pode-se considerar, então, que o fenômeno de habituação de curto prazo atua no sentido de diminuir a quantidade de neurotransmissores disponíveis na zona ativa através da diminuição das vesículas disponíveis. Quando ocorre um estímulo, a exocitose disponibiliza uma quantidade menor de glutamato. E este ajuste ocorre a uma taxa de 0,05, após o décimo estímulo.

	Em síntese: o objeto (O) desta semiose é a exocitose de glutamato. A reconfiguração orgânica na zona ativa do neurônio sensor (S) provoca uma ação de habituação de curto prazo (I) e o consequente ajuste da taxa de vesículas a menor na semiose 4, acima~\cite{kandel06, kandel00}. Isto ocorre a uma taxa de ajuste de habituação de curto prazo
	%(Thc\nomenclature{Thc}{Taxa de ajuste de habituação de curto prazo}).

	A Fig~\ref{0308s6-eps-converted-to} refere-se às características da semiose S6.

	\item \textbf{Semiose 7 - Mecanismo de habituação de longo prazo}

	Com quatro sessões de 10 estímulos separadas por algumas horas ou um dia a habituação passa de curto para longo prazo. Além da diminuição de vesículas liberadas em cada sinapse, a habituação de longo prazo também apresenta diminuição da quantidade de sinapses, ou seja, além de mudanças funcionais ocorrem alterações estruturais. Nesta situação, o número de terminais ativos diminui de 500 para 100. Para a habituação de longo prazo, que persiste por semanas, são necessárias quatro sessões de 10 estímulos separadas por algumas horas ou um dia.

	Como adiantado, os mecanismos subjacentes à habituação ainda não são totalmente reconhecidos. Portanto, por conveniência, infere-se aqui que tanto S6 quanto S7 ocorrem por ação do mesmo sinal perceptivo, a frequência com a qual o reservatório de glutamato se esvazia. Porém, cada uma delas deve apresentar rearranjos orgânicos próprios promovendo diferentes ações, no caso da habituação de curto prazo, atua na modulação da semiose S4 ajustando a menor a Qv; na habituação de longo prazo, por outro lado, a ação atua na modulação de S5 promovendo a diminuição de conexões ou Qs.

	Em síntese: certa frequência de exocitoses de glutamato
	%(Fg\nomenclature{Fg}{Frequência de exocitoses de glutamato})
	é o objeto (O) desta semiose, resultando numa certa reconfiguração orgânica (S) o que provoca uma ação de habituação de longo prazo (I) e o consequente ajuste de Qs menor na semiose 5 (I), acima~\cite{kandel06,kandel00}. Isto ocorre a uma taxa de ajuste de habituação de longo prazo
	%(Thl\nomenclature{Thl}{Taxa de ajuste de habituação de longo prazo}).

	A Fig~\ref{0309s7-eps-converted-to} refere-se às características da semiose S7.

	\figLatHere[0.6]{\eduPastaFig}{0309s7-eps-converted-to}{Características da semiose S7.}

\end{itemize}

\subsubsection*{Camada de sensibilização}
Esta camada comportamental pode ser representada heuristicamente por três semioses relevantes:

\begin{itemize}
	\item \textbf{Semiose 8 - Mecanismo de disparo do interneurônio facilitador}

	O choque na cauda provoca reações no interneurônio facilitador que conecta o neurônio sensor da cauda ao neurônio sensor do sifão modulando a ação deste último. O objeto (O) desta semiose é a presença de potencial de ação na zona ativa do interneurônio facilitador, isto provoca uma reconfiguração orgânica na zona ativa (S) resultando na exocitose de serotonina na fenda sináptica (I).

	A quantidade de serotonina é proporcional à intensidade do choque, porém como Kandel realiza o experimento com intensidade controlada e sempre igual, considera-se aqui que as exocitoses ocorrem sempre com a mesa descarga de serotonina~\cite{kandel06,kandel00, lent01}. Esta semiose é muito semelhante à S1 com a diferença que S1 ocorre no circuito mediador e S8 ocorre por ação do circuito modulatório.

	A Fig~\ref{0310s8-eps-converted-to} refere-se às características da semiose S8.

	\figLatHere[0.6]{\eduPastaFig}{0310s8-eps-converted-to}{Características da semiose S8.}

	\item \textbf{Semiose 9 - Mecanismo sensibilização de curto prazo}

	A descarga de serotonina na sinapse axoaxônica entre o interneurônio facilitador e o neurônio sensor do sifão é responsável tanto pela sensibilização de curto prazo quanto pela sensibilização de longo prazo.

	Os mecanismos subjacentes à sensibilização são mais bem conhecidos do que aqueles que se relacionam com a habituação. Esses mecanismos são postos em funcionamento por várias semioses. A serotonina (5-HT - hidroxitriptamina) liberada na sinapse é captada por dois tipos de receptores.

	O primeiro deles está associado à proteína G (Gs) que, através de sua sub-unidade alfa, ativa a enzima adenilciclase (AC) da membrana. A adenilciclase converte a adenosina trifostato (ATP) em adenosina monofostato cíclica (AMPc), aumentando assim sua concentração no terminal do neurônio sensor do sifão. Posteriormente a AMPc é desativada por ação da enzima fosfodiesterase; A AMPc ativa a proteína AMPc-dependente quinase A conectando-se à sua subunidade regulatória inibitória, deste modo liberando sua sub-unidade catalítica ativa. Esta sub unidade atua através de 3 caminhos: 1. A quinase A facilita a mobilização de vesículas de glutamato aumentando sua disponibilidade; 2. a quinase A abre canais de $Ca^{++}$ aumentando o influxo de $Ca^{++}$ prolongando o PA; e 3. a quinase A fosforila canais de potássio ($K^+$). Isto diminui a corrente $K^+$ também prolongando o PA.

	\figLatHere[0.6]{\eduPastaFig}{0311s9-eps-converted-to}{Características da semiose S9.}

	A serotonina atuando num segundo receptor conecta-se a outra proteína G (Go) que ativa a fosfolipase C (PLC) que estimula a o diacilglicerol intramembrana a ativar a proteína quinase C. A quinase C atua juntamente com a quinase A nos caminhos 1 e 2 acima.

	Conforme abordado anteriormente, a consideração de toda essa complexidade de semioses seria importante caso o interesse fosse o de reproduzir em detalhes todo o organismo da aplísia para que biólogos pudessem estudá-la sem a presença de um animal real. No entanto, como a intenção desta pesquisa é a de abstrair apenas a essência da estratégia de aprendizagem do animal, considera-se uma única semiose composta por todas as outras. Assim, a presença da serotonina (O) acarreta uma reconfiguração orgânica na zona ativa do neurônio sensor do sifão (S), o que provoca uma ação de sensibilização de curto prazo (I) e o consequente ajuste de Qv a maior na semiose S4~\cite{kandel06,kandel00}. Isto ocorre a uma taxa ajuste de sensibilização de curto prazo
	%(Tsc\nomenclature{Tsc}{Taxa de ajuste de sensibilização de curto prazo}).

	A Fig~\ref{0311s9-eps-converted-to} refere-se às características da semiose S9.

	\item \textbf{Semiose 10 - Mecanismo sensibilização de longo prazo}

	A última semiose a ser considerada (S10) é responsável pelo aumento da quantidade de terminais ativos, ou seja, ela modula a semiose S5. O mecanismo de ação para este processo também é bastante complexo iniciando-se pela presença de uma grande quantidade de serotonina.

	\figLatHere[0.6]{\eduPastaFig}{0312s10-eps-converted-to}{Características da semiose S10.}

	Quando uma alta  concentração deste neurotransmissor ocorre pela repetição do estímulo nocivo, a serotonina, parte da quinase A, gerada no interior da zona ativa do neurônio sensor do sifão, migra em direção ao núcleo do neurônio; no processo de migração da quinase A até o núcleo, ela recruta a MAP-quinase; A quinase A fosforila a proteína CREB; para ativar CREB-1, a ação repressiva de CREB-2 deve ser removida, o que ocorre por ação da MAP-quinase; CREB-1 ativa gene responsável pela ação persistente da quinase A; Outro gene é ativado pela CREB-1, responsável por acionar outro fator de transcrição, o C/EBP. Este fator conecta-se ao elemento de resposta de DNA CAAT, que ativa um terceiro gene responsável pelo crescimento de novas sinapses; RNA mensageiro e proteínas sintetizadas são enviados para todas as sinapses, porém somente a sinapse facilitada terá crescimento de novas conexões. Isto ocorre pois o RNA mensageiro é enviado em estado dormente (RNAm) aos terminais, sendo ativados pelo CPEB (proteína local autoperpetuadora) dominante transformado a partir do CPEB recessivo pela ação dos 5 pulsos de serotonina. O RNAm é responsável pelo crescimento da conexão, a CPEB pela manutenção deste crescimento.

	Em síntese: certa frequência de exocitoses de serotonina
	%(Fs\nomenclature{Fs}{Frequência de exocitoses de serotonina})
	é o objeto (O) desta semiose, resultando numa certa reconfiguração orgânica na zona ativa do neurônio sensor do sifão (S) o que provoca uma ação de sensibilização de curto prazo (I) e o consequente ajuste de Qs a maior na semiose 5, acima~\cite{kandel06,kandel00}. Isto ocorre a uma taxa ajuste de sensibilização de longo prazo
	%(Tsl\nomenclature{Tsl}{Taxa de ajuste de sensibilização de longo prazo}).

	A Fig.~\ref{0312s10-eps-converted-to} refere-se às características da semiose S10.

\end{itemize}

\subsection{Passo 5: modelagem semiótica}

A figura 3.13 representa as interações das semioses levantadas no passo anterior, resultando nas cadeias semióticas características do fenômeno biológico em estudo. Trata-se de uma representação gráfica da estratégia de aprendizado abstraída dos fenômenos de habituação e sensibilização da aplísia, através da aplicação de conceitos semióticos. Essas interações estão organizadas em diferentes camadas de subordinação conforme levantamento realizado no passo 2 do método proposto e retrata os processos abstraídos segundo o nível focal celular e de suas organelas definido no passo 3 do MTS. Seguem abaixo as descrições das interações semióticas em cada camada de subordinação.

\paragraph*{Camada de comportamento normal}

Uma primeira semiose (S1), na zona de disparo do neurônio sensor do sifão, é responsável por interpretar se o toque do sifão da aplísia é suficientemente forte para gerar um potencial de ação, trata-se de uma interpretação binária (discreta). Quando o evento (toque) é aceito como válido rompendo o limiar de -55 mV (objeto de S1), ocorre um rearranjo iônico na zona de disparo (signo de S1) resultando num potencial de ação (interpretante de S1); o potencial de ação gerado na zona de disparo chega à zona ativa do mesmo neurônio (objeto de S2) e desencadeia uma segunda semiose (S2), ocorrendo, então um rearranjo na concentração de íons $Ca^{++}$ no interior da zona ativa (signo de S2) resultando na exocitose de glutamato na fenda sináptica (interpretante de S2). A presença de glutamato na fenda sináptica (objeto de S3) provoca uma nova configuração orgânica no neurônio motor da guelra (signo de S3) que, por sua vez, provoca reações físicas específicas (interpretante de S3) no sistema de enervação da guelra provocando sua retração.

A exocitose de glutamato esvazia as vesículas mobilizadas na zona ativa do neurônio sensor do sifão, então o sistema procura restabelecer tais vesículas através do processo de endocitose. Abstrai-se, então, que uma nova semiose (S4) ocorre na zona ativa do mesmo neurônio. O sinal perceptivo de ausência de vesículas (objeto de S4), provoca uma nova configuração orgânica (signo de S4), cujo resultado é a endocitose que ocorre na zona ativa do neurônio sensor do sifão (interpretante de S4). Este processo modula a quantidade de vesículas disponíveis em S2 (Qv) a maior ou a menor (efeito somatório) de acordo com a modulação do sistema em comportamento normal, habituado ou sensibilizado.

Quando a frequência de exocitoses atinge certo valor, ocorre uma mudança na concentração proteica (objeto de S5) no interior da zona ativa do neurônio sensor do sifão resultando numa nova semiose (S5), o que provoca uma certa configuração orgânica específica (signo de S5) que proporciona o aumento ou diminuição de conexões sinápticas (Qs) (interpretante de S5). Isto modula a quantidade de conexões sinápticas, a maior ou a menor, em S2, provocando um efeito multiplicador na quantidade de vesículas disponíveis.

\paragraph*{Camada de habituação}

A camada de habituação apresenta duas semioses, S6 e S7, responsáveis pelas habituações de curto e longo prazos.

A semiose S6 tem por objeto o sinal perceptivo de reservatório vazio (exocitose), da mesma forma que S4. Isto provoca uma configuração orgânica específica na zona ativa (signo de S6) de tal forma que S6 promove uma ação de habituação de curto prazo (interpretante de S6). Esta ação modula a semiose S4 através do que pode ser chamado de taxa de ajuste de habituação de curto prazo (Thc).  Esta taxa atualiza o estado de S4 para uma condição de menor número de vesículas disponíveis e consequentemente menor quantidade de glutamato liberado na fenda sináptica quando ocorrer S2 novamente. Isto é necessário para se instalar o estado de habituação de curto prazo. Em resumo, abstrai-se que, sempre que ocorre uma exocitose na zona ativa do neurônio sensor do sifão (o que corresponde a toques no sifão), S6 a contabiliza e ajusta a próxima endocitose a repor uma quantidade menor de vesículas, o que acarreta uma próxima exocitose mais fraca, habituando o sistema.

A semiose S7 ocorre através de um mecanismo próprio tendo como sinal perceptivo certa frequência de exocitoses (objeto de S7). Assim, procede também a uma reconfiguração orgânica na zona ativa do neurônio sensor do sifão (signo de S7), desta feita correspondente a uma habituação de longo prazo. O resultado é uma ação de habituação de longo prazo (interpretante de S7).

Esta ação modula a semiose S5 para uma condição de menor quantidade de conexões entre neurônio sensor do sifão e neurônio motor da guelra. Isto também acarreta uma menor descarga de glutamato na fenda sináptica, porém o efeito de tal ajuste dura um tempo bem maior que o conseguido através de S6. A modulação de S5 por S7 ocorre a uma taxa de ajuste habituação de longo prazo (Thl).

\paragraph*{Camada de sensibilização}

A camada de sensibilização representa um processo desencadeado por um circuito diferente, chamado de circuito modulatório. Uma semiose, S8, muito semelhante à semiose S1, é ativada pelo choque na cauda da aplísia. Ao romper o limiar de -55 mV (objeto de S8) na zona de disparo do interneurônio facilitador ocorre um rearranjo iônico compatível com a intensidade do estímulo (signo de S8) responsável por uma exocitose de serotonina na fenda sináptica entre o interneurônio facilitador da cauda e o neurônio sensor do sifão (interpretante de S8).

A concentração de serotonina (signo de S9) inicia uma nova semiose na zona ativa do neurônio sensor do sifão (S9), provocando uma nova configuração orgânica nesta região (signo de S9), o que acarreta uma ação de sensibilização de curto prazo (interpretante de S9). Isto acaba modulando, a maior, a semiose S4 numa taxa específica chamada aqui de taxa de ajuste de sensibilização de curto prazo (Tsc). Assim, na próxima ocorrência de S4, a reposição das vesículas de glutamato disponibiliza uma quantidade muito maior de neurotransmissor, maior ainda do que a quantidade envolvida no comportamento normal. Isto modula o sistema para um estado de sensibilização de curto prazo.

Finalmente, a semiose S10 é instalada quando o sistema atinge certa frequência de exocitoses de serotonina (objeto de S10), devido ao treinamento de sensibilização de longo prazo. Esse excesso de serotonina provoca uma reconfiguração orgânica específica para a sensibilização de longo prazo (signo de S11), provocando a ação de sensibilização de longo prazo (interpretante de S10) e o consequente aumento de conexões sinápticas entre o neurônio sensor do sifão e o sensor motor da guelra. Isto acaba por modular a Semiose S5, a maior, segundo a taxa de ajuste de sensibilização de longo prazo (Tsl).

\figLatHere[0.9]{\eduPastaFig}{0313modelosemiotico-eps-converted-to}{Diagrama representando o modelo semiótico abstraído do comportamento de habituação e sensibilização do \textit{Aplysia californica}.}

\section{Do modelo semiótico à UBA-HS: passo 6 do MTS}

Codificação da UBA-HS usando a linguagem de especificação UBA.

\section{Aplicação (didática): busca de texto}

\subsection{O problema}

O objetivo desta aplicação é investigar, nos quatro arquivos citados, linha por linha, a existência de passagens nas quais personagens perambulam pelas ruas de Dublin e que também apresentem referências aos locais por onde passam. O andar pela cidade foi considerado como o ``tema'' da busca, correspondendo ao evento mediador do metamodelo e representado por verbos que indiquem as ações das personagens: andar, cruzar (ruas, avenidas, etc.), correr, olhar (monumentos, paisagens, etc.), dentre outros. Os locais da cidade, ruas, parques, pontes e outros, foram  considerados como o ``contexto'' da busca, correspondendo ao evento modulador.

O aplicativo inicia a busca de tema e contexto, linha por linha, em ``modo normal''. Quando um primeiro tema é encontrado, uma ``ação normal'' é tomada, o que significa acionar a verificação de contexto. Se o contexto não for encontrado na mesma linha, e isto se repetir nas linhas que se seguem, o sistema tende ao  ``modo habituado'' até começar a ignorar a verificação de contexto em linhas consequentes mesmo que contenham o tema. Este efeito permanece por algum tempo, com o sistema retornando aos poucos ao modo normal. Em termos práticos, significa que trechos habituados, apesar de apresentarem o tema, provavelmente estão fora de contexto e o aplicativo ``ignora'' (``ação habituada'') por algum tempo a busca completa.

Quando, ao contrário, o contexto é encontrado com frequência em linhas que apresentam o tema, o sistema tende ao modo ``sensibilizado'' até que uma ``ação sensibilizada'' seja tomada, o que significa buscar detalhes do contexto em linhas que contenham o tema. Esses detalhes são aqui entendidos como substantivos próprios referentes à cidade de Dublin como \textit{Liffey} e \textit{Westland}. Esses substantivos próprios foram colhidos no trabalho de Gunn e Hart~\cite{gunn04}. Quando sensibilizado, o sistema permanece assim por algum tempo retornando ao modo normal à medida que a busca continua.

\subsection{Aplicação}

	\chapter{Tradução gênica: UBA inspirada em comportamento genético (UBA-TG)}

Em genética, a proteína representa o fenótipo de um indivíduo ou de uma espécie, ou seja, seu metabolismo e comportamento. O ácido desoxirribonucleico (DNA), por sua vez, representa o genótipo, conjunto de informações responsáveis pela replicação e procriação. O ácido ribonucleico (RNA) é a substância química que liga os dois mundos: informação e expressão. Finalmente, tradução gênica é o nome dado ao processo que, partindo da informação genética, fabrica as proteínas necessárias à vida~\cite{ridley99}.

Na tradução gênica, uma fita de RNA mensageiro (RNAm) é produzida no núcleo celular e migra ao citoplasma onde é acolhida pelo ribossomo. O ribossomo é uma estrutura bipartida composta de RNA ribossômico (RNAr) que atua como uma máquina de tradução/fabricação. Primeiro, o ribossomo encontra o início da mensagem na fita de RNAm, então convoca uma outra substância também presente no citoplasma,o RNA transportador (RNAt), que será responsável por levar até o ribossomo o aminoácido que inicia a tradução da mensagem do gene. Após o início do processo, uma cadeia de aminoácidos é formada pela convocação de outros RNAt que transportam  moléculas de aminoácidos específicas, segundo a especificação da fita de RNAm~\cite{ridley99}.

\section{Modelagem semiótica da UBA-TG: passos 1 ao 5 do MTS}

\subsection{Passo 1: análise preliminar}

O trabalho da máquina ribossômica de tradução/fabricação ocorre de maneira precisa e rápida. Percebe a presença de RNAm e, num primeiro momento, trabalha com o objetivo de encontrar o início da mensagem. Num segundo momento, ajusta seu comportamento para a tradução/fabricação propriamente dita.

A tradução gênica, com sua dinâmica e seus dois comportamentos, reconhecimento e tradução/fabricação, aponta para aplicações potenciais em diversos campos como os da criptografia, segurança e privacidade de redes e reconhecimento de linguagem natural entre outros.


\subsection{Passo 2: definição da arquitetura de subordinação}

Duas camadas de subordinação podem ser abstraídas do fenômeno de tradução gênica. Tendo em vista o comportamento do ribossomo, uma primeira camada refere-se às atividade de percepção e busca. Nesta camada, o ribossomo permanece em constante estado de atenção até perceber a presença de uma nova fita a ser traduzida. Ao percebê-la, busca o início da mensagem.

A segunda camada é responsável pela tradução/fabricação propriamente dita. A camada de busca seria hierarquicamente superior à camada de percepção, pois, iniciado o processo de tradução/fabricação, ele não será interrompido pela presença de nenhuma nova fita até que o ribossomo encontre o final da mensagem. %(Figura 3.1).

\figLatHere{\eduPastaFig}{ubatgsubordinacao}{Arquitetura de subordinação abstraída do fenômeno de tradução gênica.}

\subsection{Passo 3: definição do nível focal}

O nível focal adequado ao estudo da transposição semiótica do fenômeno de tradução gênica é o nível das organelas celulares. Consequentemente, o nível inferior ou micro-semiótico, iniciador dos processos, é o nível genético das moléculas de RNA e o nível superior ou macro-semiótico, que apresenta as restrições naturais, é o nível celular %(Figura 3.2).

\figLatHere[0.6]{\eduPastaFig}{ubatgnivelfocal}{Representação dos níveis hierárquicos abstraídos do fenômeno de tradução gênica.}

\subsection{Passo 4: levantamento das semioses relevantes}

Em genética, códon é uma sequência de três bases hidrogenadas. No RNA essas bases podem ser Adenina (A), Guanina (G), Uracila (U) e Citosina (C). Uma fita de RNAm contém uma sequência de códons a ser traduzida pelo ribossomo. O ribossomo, dividido em duas subpartes, move-se ao longo da fita de RNAm traduzindo cada códon encontrado em um aminoácido específico dentre vinte verificáveis. Ao ler certo códon, o ribossomo convoca o RNAt capaz de se acoplar a ele através de seu anticódon, sendo que, os acoplamentos possíveis são A com U e G com C. Cada RNAt convocado transporta um aminoácido específico. Após a leitura de todos os códons, forma-se uma proteína, ou seja, uma cadeia de aminoácidos dobrada em uma forma distinta de acordo com sua sequência.

Dentre os vinte  códons, a sequência AUG indica o início da mensagem a ser traduzida, e as sequências UAA, UAG e UGA indicam o final da mensagem. Assim, o códon inicial (AUG) só poderá se acoplar a um RNAt que contenha o anticódon UAC~\cite{ridley99,lodish12}.

\subsubsection*{Camada de percepção e busca}

Esta camada comportamental pode ser representada heuristicamente por duas semioses relevantes: S1, S2. S1 seria a semiose de percepção da fita de RNAm e a S2 a semiose de busca pelo início da mensagem. Enquanto S1 ocorre apenas uma vez, S2 se repete até que o início seja encontrado. Ambas semioses ocorrem na subparte menor do ribossomo.

\begin{itemize}
	\item \textbf{Semiose 1 - Percepção da fita de RNAm.}

		Mediante a necessidade celular em fabricar certa proteína, a presença de uma fita de RNAm provoca o início do processo de tradução gênica. Neste instante, ocorre a conexão da subparte menor do ribossomo com o sítio de ligação da fita de RNAm (\textit{Ribossome Bindind Site - RBS}). Após essa ligação, a subparte menor do ribossomo acopla-se a uma molécula de RNAt que carrega uma unidade do aminoácido metionina. O RNAt convocado deve apresentar códon AUG, necessário para acoplamento ao códon UAC da fita de RNAm que é indicativo do início da mensagem.

		Em termos semióticos (tríade Objeto/Signo/Interpretante) tem-se: a fita de RNAm como o sinal perceptivo que designa o objeto (O), a ligação RBS/Ribossomo como signo (S) e o acoplamento entre a subparte menor do ribossomo e a molécula de RNAt (AUG) como interpretante (I).

		\figLatHere[0.6]{\eduPastaFig}{ubatgs1-crop}{Características da semiose S1.}

		\item \textbf{Semiose 2 - Busca do início da mensagem.}

		Após o acoplamento entre a subparte menor do ribossomo e o RNAt (AUG), o ribossomo se desloca para o primeiro códon, então dois resultados são possíveis: se este códon for do tipo UAC, ele se acopla ao RNAt (AUG) produzindo a metionina na subparte maior do ribossomo e o sistema passa para a semiose S3 na camada de tradução/fabricação, caso contrário, o ribossomo se desloca para o códon seguinte realizando nova S2, permanecendo na camada de percepção e busca até encontrar o início da mensagem.

		Em termos semióticos (tríade Objeto/Signo/Interpretante) tem-se: o acoplamento ribossomo/RNAt (AUG) como o sinal perceptivo que designa o objeto (O), o posicionamento sobre o códon de leitura como signo (S) e o resultado da ação (acoplamento ou não) como interpretante (I).

		\figLatHere[0.6]{\eduPastaFig}{ubatgs2-crop}{Características da semiose S2.}

\end{itemize}

\subsubsection*{Camada de tradução/produção}

Esta camada comportamental também pode ser representada por duas semioses: S3 é responsável pela leitura dos códons seguintes e consequente encadeamento dos aminoácidos que formarão a proteína requisitada (S3 ocorre de forma recursiva até o códon final), e S4 é responsável pelo fechamento da cadeia de aminoácidos.

\begin{itemize}
	\item \textbf{Semiose 3 - Formação da cadeia de aminoácidos}

	Após a identificação do início da mensagem contida na fita de RNAm e a consequente produção da metanina, o ribossomo se posiciona sobre o próximo códon e produz o aminoácido correspondente. O ribossomo reproduz esse comportamento de forma recursiva até que encontre um dos seguintes códons de fechamento: UAA, UAG ou UGA.

	Em termos semióticos (tríade Objeto/Signo/Interpretante) tem-se: a metionina como sinal perceptivo que designa o objeto (O), o posicionamento sobre o próximo códon de leitura como signo (S) e a fabricação de uma nova enzima como (I).

	\figLatHere[0.6]{\eduPastaFig}{ubatgs3-crop}{Características da semiose S3.}

	\item \textbf{Semiose 4 - Fechamento da cadeia enzimática}

	Ao encontrar um dos códons de fechamento (UAA, UAG ou UGA) o ribossomo desliga a fabricação de proteína sem agregar nenhum novo aminácido.

	Em termos semióticos (tríade Objeto/Signo/Interpretante) tem-se: o último aminoácido domo o objeto (O), o posicionamento sobre o próximo códon de fechamento como signo (S) e a finalização do processo como interpretante (I).

	\figLatHere[0.6]{\eduPastaFig}{ubatgs4-crop}{Características da semiose S4.}

\end{itemize}

\subsection{Passo 5: modelagem semiótica}

Para que uma célula possa fabricar as proteínas necessárias ao organismo, o RMAm formado no núcleo da célula migra para o citoplasma. O encontro da fita de RNAm com um ribossomo inicia o processo de tradução gênica que vai sintetizar proteínas específicas segundo o código contido na fita de RNAm.

\paragraph*{Camada de percepção e busca}

O ribossomo encontra a fita de RNAm (objeto de S1) conectando sua sub-parte menor ao sítio de ligação RBS (signo de S1). Devido às características intrínsecas ao fenômeno, o resultado deste encontro é a conexão do RNAt que transporta o aminoácido metionina (interpretante de S1 e objeto de S2). O sistema inicia a procura pelo início da mensagem posisionado-se sobre o primeiro códon (signo de S2). Há dois interpretantes possíveis para S2: se o primeiro códon for UAC existe a conexão com a inserção do primeiro aminoácido (metionina) (objeto de S3) e a cadeia semiótica avança para S3, caso contrário, o sistema se posiciona sobre o segundo códon e S2 se repete e, continua a se repetir códon após códon até que ocorra a conexão AUG/UAC.

\paragraph*{Camada de tradução/produção}

O encontro do início da mensagem com a conexão AUG/UAC e a inserção da metionina (objeto de S3) tem como signo o posicionamento sobre o códon seguinte (signo de S3) e a consequente inserção do aminoácido correspondente (interpretante de S3). A semiose S3 se repete com objetos específicos segundo os códons encontrados, formando a cadeia de aminoácidos. A produção de certo aminoácido (objeto de S4) faz com que o sistema se posicione sobre um códon do tipo UAA, UAG ou UGA (signo de S4) que indica o final do processo. O interpretante de S4 é a finalização da tradução/fabricação.

\figLatTop[0.8]{\eduPastaFig}{ubatgmodelosemiotico-crop.eps}{Diagrama representando o modelo semiótico abstraído do comportamento tradução gênica.}

\section{Do modelo semiótico à UBA-TG: passo 6 do MTS}

Codificação da UBA-TG usando a linguagem de especificação UBA.

\section{Aplicação (didática): atendimento do Poupa Tempo }

\subsection{O problema}

Apresentar o problema

\subsection{Aplicação}

	\chapter{Cadeia de vacância: UBA inspirada em comportamento de espécie (UBA-CV)}

Em economia, cadeia de vacância é o nome dado ao conjunto de trocas sequenciais que beneficiam vários indivíduos sucessivamente. Como exemplo, no mercado imobiliário, a aquisição de uma casa nova por certo indivíduo propagará uma série de trocas secundárias que tendem a melhorar a situação de moradia de outros indivíduos. Os recursos que são afetados por esse fenômeno apresentam certas características em comum: devem ser desejados e relativamente difíceis de serem conseguidos, só podem ser ocupados por um único indivíduo de cada vez e só podem ser ocupados se estiverem vazios~\cite{chase12}.

Este tipo de fenômeno é também percebido no comportamento de alguns animais, dentre os quais, o \textit{Pagurus longicarpus}, uma espécie de caranguejo comum na costa leste dos EUA. Esses animais utilizam conchas como abrigos e as carregam consigo. Conforme crescem, os paguros procuram abrigos maiores e melhores abandonando suas antigas conchas que são utilizadas por animais mais novos e menores desencadeando uma série de trocas vantajosas para todos~\cite{chase12}.

Descobriu-se recentemente que os paguros usam dois tipos de cadeia. Na cadeia assíncrona apenas um caranguejo por vez encontra uma nova concha, na cadeia síncrona vários animais fazem fila, por ordem de tamanho, atrás do indivíduo que estiver examinando a concha vazia. Neste segundo caso, quando o primeiro caranguejo da fila se acomoda numa nova concha, o próximo ocupa a concha abandonada~\cite{chase12}.

\section{Modelagem semiótica da UBA-CV: passos 1 ao 5 do MTS}

\subsection{Passo 1: análise preliminar}

O estudo da cadeia de vacância no campo biológico pode trazer inúmeros \textit{insights} para soluções de problemas humanos, desde o ajuste de oferta e demanda no mercado imobiliário até a simples organização de filas de atendimento em corporações dos setores público e privado.

O comportamento adaptativo dos paguros fica mais claro na modalidade síncrona das trocas de conchas. Mediante a vacância percebida numa concha atraente (intacta e de bom tamanho), o que se esperaria de um animal de cérebro relativamente pequeno e simples seria a competição acirrada pelo novo abrigo. No entanto, o que se observa é um comportamento orquestrado, sugerindo a presença de cognição social sofisticada.

No caso dos paguros, existe um único fator (tamanho) responsável pela organização, mas nada impede de serem considerados conjuntos de fatores como organizadores da cadeia de vacância. Por exemplo, o atendimento preferencial em bancos e outras organizações, pode ser resolvido aplicando-se não apenas o critério de idade e outras características perceptíveis, mas também outros fatores não aparentes e que podem constar como informação relevante nos prontuários dos clientes.

\subsection{Passo 2: definição da arquitetura de subordinação}

No caso do comportamento assíncrono, apenas uma camada comportamental está presente: a busca e ocupação ocorrem de maneira serial. Mas, no comportamento síncrono, entre a busca e a ocupação, surge uma camada de espera em fila. Portanto, será considerado aqui o comportamento síncrono e suas duas camadas: busca/ocupação e posicionamento em fila.

\figLatHere[0.6]{\eduPastaFig}{ubacvsubordinacao-crop}{Arquitetura de subordinação abstraída do comportamento dos paguros.}

\subsection{Passo 3: definição do nível focal}

O nível focal adequado ao estudo da cadeia de vacância no comportamento dos paguros é o nível do organismo ou do indivíduo. Consequentemente, o nível inferior ou micro-semiótico, iniciador dos processos, é o nível dos processos neurais e o nível superior ou macro-semiótico, que apresenta as restrições naturais, é o ecológico %(Figura 3.2).

\figLatHere[0.7]{\eduPastaFig}{ubacvnivelfocal}{Representação dos níveis hierárquicos abstraídos do comportamento dos paguros.}

\subsection{Passo 4: levantamento das semioses relevantes}


Diferentemente dos casos anteriores, UBA-HS e UBA-TG,as pesquisas relativas à cadeia de vacância no comportamento dos paguros não leva em conta os processos internos, neurais ou celulares. O paguro seria, então, uma espécie de caixa preta e apenas seu comportamento como indivíduo é registrado. Assim, as representações decorrentes dos processos semióticos não serão especificadas, apenas consideradas como existentes.

Em resumo, no comportamento síncrono, o indivíduo se desloca em seu meio-ambiente e, ao detectar uma concha atraente abandonada, é impelido a uma troca, abandonando a concha atual e tomando para si o novo abrigo. No entanto, a presença de outros indivíduos faz com que o animal em questão se posicione em fila de acordo com seu tamanho~\cite{chase12}.

\subsubsection*{Camada de busca e ocupação}

Esta camada comportamental pode ser representada heuristicamente por três semioses relevantes: S1, S2 e S3. A semiose S1 seria a percepção da concha vazia, S2 a percepção da não existência (ou não existência) de concorrentes e S3 a ocupação da nova concha. Nota-se que , se o interpretante de S2 for a não existência de outros indivíduos, o percurso direto S1, S2 e S3 representaria uma troca assíncrona.

\begin{itemize}
	\item \textbf{Semiose 1 - Percepção nova concha.}

	Com o crescimento do animal, a concha em que habita se torna pequena demais. A percepção de uma nova concha provoca o início do processo da cadeia de vacância e o paguro se predispõe a ocupar da nova concha.

	Em termos semióticos (tríade Objeto/Signo/Interpretante) tem-se: a percepção de uma nova concha como o sinal perceptivo que designa o objeto (O), signo (S) não verificável e a predisposição para a ocupação como interpretante (I).

	\figLatHere[0.6]{\eduPastaFig}{ubacvs1-crop}{Características da semiose S1.}

	\item \textbf{Semiose 2 - Existência ou não de outros indivíduos.}

	O animal predisposto a ocupar a nova concha deve verificar se há outros indivíduos na mesma situação.

	Em termos semióticos (tríade Objeto/Signo/Interpretante) tem-se: predisposição à ocupação como o sinal perceptivo que designa o objeto (O), signo (S) não verificável e o resultado da verificação (há outros indivíduos ou não) como interpretante (I).

		\figLatHere[0.6]{\eduPastaFig}{ubacvs2-crop}{Características da semiose S2.}

	\item \textbf{Semiose 3 - Ocupação da nova concha.}

	Quando não existe outros indivíduos maiores do que o animal em questão, ocorre a ocupação

	Em termos semióticos (tríade Objeto/Signo/Interpretante) tem-se: Não existência de outros animais maiores como objeto (O), signo (S) não verificável e a efetivação da ocupação como interpretante (I).

			\figLatHere[0.6]{\eduPastaFig}{ubacvs3-crop}{Características da semiose S3.}

\end{itemize}

\subsubsection*{Camada de posicionamento em fila}

Esta camada comportamental é representada por duas semioses: S4 é responsável pela percepção dos diferentes tamanhos do indivíduos envolvidos e consequente posicionamento do animal em questão, e S5 é responsável pelo retorno à camada de ocupação quando o animal em questão percebe não haver mais nenhum indivíduo maior do que ele.

\begin{itemize}
	\item \textbf{Semiose 4 - Posicionamento em fila}

	A percepção de outros indivíduos no processo da cadeia de vacância faz o animal em questão se posicionar em fila segundo seu tamanho. Enquanto forem percebidos indivíduos maiores do que o animal em questão, a semiose S4 se repete.

	Em termos semióticos (tríade Objeto/Signo/Interpretante) tem-se: presença de outros indivíduos como sinal perceptivo que designa o objeto (O), signo (S) não verificável e o posicionamento em fila (I).

			\figLatHere[0.6]{\eduPastaFig}{ubacvs4-crop}{Características da semiose S4.}


	\item \textbf{Semiose 5 - Retorno à camada de busca e ocupação}

	Ao não perceber indivíduos maiores, o animal em questão, finalmente conclui a ocupação.

	Em termos semióticos (tríade Objeto/Signo/Interpretante) tem-se: não percepção de indivíduos maiores como objeto (O), signo (S) retorno à camada de busca e ocupação como interpretante (I).


			\figLatHere[0.6]{\eduPastaFig}{ubacvs5-crop}{Características da semiose S5.}

\end{itemize}

\subsection{Passo 5: modelagem semiótica}

Um animal específico tem a necessidade de encontrar uma nova concha mais adequada ao seu tamanho. Abaixo, segu a descrição com a omissão dos signos pois, como já declarado, não são verificáveis.

\subsubsection*{Camada de busca e ocupação}

Ao encontrar uma nova concha (objeto de S1), o paguro predispõe-se à troca (interpretante de S1). A predisposição (objeto de S2) provoca a verificação da presença ou ausência de outros indivíduos (interpretante de S2). Se não houver outros indivíduos (objeto de S3), a ocupação ocorre (interpretante de S3). Se houver outros indivíduos, o sistema altera para a camada de posicionamento em fila provocando S4.

\subsubsection*{Camada de posicionamento em fila}

A presença de outros indivíduos (objeto de S4) faz o animal em questão postar-se em fila por ordem de tamanho (interpretante de S4). Enquanto houver indivíduos maiores, S4 se repete. Ao verificar a não existência de outros indivíduos (objeto de S5) o sistema retorna à camada de ocupação (interpretante de S5)

\figLatTop[0.8]{\eduPastaFig}{ubacvmodelosemiotico-crop}{Diagrama representando o modelo semiótico abstraído da cadeia de vacância do comportamento do \textit{Pagurus longicarpus}.}

\section{Do modelo semiótico à UBA-CV: passo 6 do MTS}

Codificação da UBA-CV usando a linguagem de especificação UBA.

\section{Aplicação (didática): atendimento médico com prioridade }

\subsection{O problema}

Descrever o problema.

\subsection{Aplicação}


\begin{partbacktext}
\part{Resultados}
\noindent Reflexão a respeito das potenciais aplicações e a criação de um catálogo de UBAs.
\end{partbacktext}

	\chapter{Usos dos Resultados (?)}
	\label{cap:resultados} % Always give a unique label


%\motto{Use the template \emph{chapter.tex} to style the various elements of your chapter content.}
\chapter{Conclusão}
\label{cap:conclusao} % Always give a unique label
% use \chaptermark{}
% to alter or adjust the chapter heading in the running head

% A seção de conclusão de cada capítulo prevê um balanço dos resultados obtidos como produto final do desenvolvimento das técnicas propostas e exercitadas no capítulo. Tratando-se de uma área inovadora, são raras as situações em que o trabalho associado aos assuntos tratados no capítulo não apresente novidades. Assim, espera-se que na conclusão do capítulo seja feito um levantamento de todas as inovações, e também de todas as possíveis contribuições que o conteúdo do capítulo possa ter trazido para as áreas de interesse a que se referem. Ideias novas que possam ter surgido devem ser indicadas como temas para trabalhos futuros, ou para possíveis publicações adicionais a serem desenvolvidas. Ensaios que tenham produzido resultados negativos devem ser analisados, e para as avaliações negativas, devem ser apresentadas justificativas ou alternativas para a correção ou melhoria do seu desempenho. Muito útil é terminar a conclusão do capítulo com um quadro de análise dos pontos altos e baixos identificados, com indicações objetivas de formas de correção ou de melhoramentos propostos.

	Esta monografia foi desenvolvida sob a luz da multidisciplinaridade, reunindo articulações entre três áreas de conhecimento, a biologia, a computação e, como elemento de ligação, a semiótica, dando origem à convergência ente computação bioinspirada e a teoria dos autômatos adaptativos.

O \textbf{Método de Transposição Semiótica} - fundamentado em outros três corpos teóricos, dois deles de base semiótica, a teoria da significação de Uexküll e o estruturalismo hierárquico de Salthe, e o outro, baseado na nova IA, mais precisamente, na arquitetura de subordinação de Brooks - foi utilizado como elemento de ligação entre os campos biológico e computacional. Este método, apresenta seis passos que o estabelecem como um método geral. Esses passos, em seus devidos momentos, estabeleceram técnicas especiais baseadas em recortes teóricos específicos extraídos dos trabalhos de Uexküll, Salthe e Brooks.

A escolha dos comportamentos de habituação e sensibilização da aplísia, apresentados por Kandel, mostraram-se bastante adequados como estudo de caso, primeiramente, pela precisão e clareza com que Kandel descreve suas descobertas no campo da neurociência e, segundo, por proporcionar um recorte de dimensões tratáveis no tempo relativamente curto da pesquisa.

Tendo em vista que a proposta do MTS é de abstrair somente os mecanismos essenciais subjacentes ao algoritmo biológico em estudo, e não a complexidade orgânica total do animal, conclui-se que uma importante contribuição do método à modelagem computacional biomimética vem justamente de seu caráter sintético, que promove operações heurísticas auxiliando o direcionamento do olhar do pesquisador na busca dos processos sígnicos realmente imprescindíveis à transposição entre os campos biológico e computacional.

Outra contribuição a ser destacada é que o MTS se apresentou como uma ferramenta de modelagem eficaz para o seu propósito. Os modelos específicos resultantes não decorreram de simples reduções, mas uma síntese fundamentada que resultou num modelo semiótico e em seu consequente metamodelo computacional.

Além dessas duas contribuições gerais, outra de caráter mais específico pode ser descrita. Com referência ao aplicativo-exemplo, pode-se concluir que a aprendizagem baseada em habituação e sensibilização se mostrou um mecanismo interessante na implementação de buscas contextuais. Podem-se imaginar sistemas de busca em grandes volumes de dados nos quais o ajuste dos parâmetros resultem em diferentes resultados de acordo com a necessidade do usuário. Quando o tempo de busca for questão crítica, ajusta-se o sistema para tender à habituação e entregar um resultado que, mesmo não refletindo a totalidade da informação útil, consegue entregar uma parte significativa num tempo reduzido.

Além da aplicação do MTS em si, esta monografia mostrou sua MTS em conjunto com dispositivos adaptativos para a elaboração de especificações adaptativas no sentido proposto por Vega[11].

Partindo de fenômenos biológicos, por exemplo, pelo animal Aplysia, o emprego das semioses do MTS produziu um modelo comportamental que combinou estados de habituação e de sensibilização.

Estes são estados que modelam a reação do animal a diferentes tipos e intensidades de sinais por ele percebidos. Um dispositivo adaptativo foi utilizado para modelar os particulares fenômenos que desencadeiam as mudanças comportamentais do animal. A técnica de especificações adaptativas suportou a representação do modelo adaptativo assim elaborado. Um posterior refinamento originou um padrão de comportamento chamado de unidade biomimética adaptativa habituação-sensibilização (UBA-HS), também proposto nesta monografia.

Estes resultados apontam na direção de um catálogo de padrões biomiméticos adaptativos que se encontra em elaboração nesta pesquisa. Em particular, uma versão preliminar da UBA-HS foi utilizada em um ensaio de modelagem do KWIC [13] e está em estudo o seu emprego no contexto da Internet das Coisas [14], os quais não foram alvo de discussão detalhada neste trabalho.


\appendix
%\motto{All's well that ends well}
%\chapter{Chapter Heading}
%\label{introA} % Always give a unique label
% use \chaptermark{}
% to alter or adjust the chapter heading in the running head

%Use the template \emph{appendix.tex} together with the Springer document class SVMono (monograph-type books) or SVMult (edited books) to style appendix of your book in the Springer layout.


%\section{Section Heading}
%\label{sec:A1}
% Always give a unique label
% and use \ref{<label>} for cross-references
% and \cite{<label>} for bibliographic references
% use \sectionmark{}
% to alter or adjust the section heading in the running head
%Instead of simply listing headings of different levels we recommend to let every heading be followed by at least a short passage of text. Furtheron please use the \LaTeX\ automatism for all your cross-references and citations.


%\subsection{Subsection Heading}
%\label{sec:A2}
%Instead of simply listing headings of different levels we recommend to let every heading be followed by at least a short passage of text. Furtheron please use the \LaTeX\ automatism for all your cross-references and citations as has already been described in Sect.~\ref{sec:A1}.

%For multiline equations we recommend to use the \verb|eqnarray| environment.
%\begin{eqnarray}
%\vec{a}\times\vec{b}=\vec{c} \nonumber\\
%\vec{a}\times\vec{b}=\vec{c}
%\label{eq:A01}
%\end{eqnarray}

%\subsubsection{Subsubsection Heading}
%Instead of simply listing headings of different levels we recommend to let every heading be followed by at least a short passage of text. Furtheron please use the \LaTeX\ automatism for all your cross-references and citations as has already been described in Sect.~\ref{sec:A2}.

%Please note that the first line of text that follows a heading is not indented, whereas the first lines of all subsequent paragraphs are.

% For figures use
%
%\begin{figure}[t]
%\sidecaption[t]
%\centering
% Use the relevant command for your figure-insertion program
% to insert the figure file.
% For example, with the option graphics use
%\includegraphics[scale=.65]{fig/raw/figure}
%
% If not, use
%\picplace{5cm}{2cm} % Give the correct figure height and width in cm
%
%\caption{Please write your figure caption here}
%\label{fig:A1}       % Give a unique %label
%\end{figure}

% For tables use
%
%\begin{table}
%\caption{Please write your table caption here}
%\label{tab:A1}       % Give a unique label
%
% For LaTeX tables use
%
%\begin{tabular}{p{2cm}p{2.4cm}p{2cm}p{4.9cm}}
%\hline\noalign{\smallskip}
%Classes & Subclass & Length & Action Mechanism  \\
%\noalign{\smallskip}\hline\noalign{\smallskip}
%Translation & mRNA$^a$  & 22 (19--25) & Translation repression, mRNA cleavage\\
%Translation & mRNA cleavage & 21 & mRNA cleavage\\
%Translation & mRNA  & 21--22 & mRNA cleavage\\
%Translation & mRNA  & 24--26 & Histone and DNA Modification\\
%\noalign{\smallskip}\hline\noalign{\smallskip}
%\end{tabular}
%$^a$ Table foot note (with superscript)
%\end{table}
%


\backmatter

	%\Extrachap{Glossário}

%Use the template \emph{glossary.tex} together with the Springer document class SVMono (monograph-type books) or SVMult (edited books) to style your glossary\index{glossary} in the Springer layout.

%\runinhead{glossary term} Write here the description of the glossary term. Write here the description of the glossary term. Write here the description of the glossary term.

%\runinhead{glossary term} Write here the description of the glossary term. Write here the description of the glossary term. Write here the description of the glossary term.

%\runinhead{glossary term} Write here the description of the glossary term. Write here the description of the glossary term. Write here the description of the glossary term.

%\runinhead{glossary term} Write here the description of the glossary term. Write here the description of the glossary term. Write here the description of the glossary term.

%\runinhead{glossary term} Write here the description of the glossary term. Write here the description of the glossary term. Write here the description of the glossary term.
	%\Extrachap{Soluções}

%\section*{Problemas do Capítulo~\ref{intro}}

%\begin{sol}{prob1}
%The solution\index{problems}\index{solutions} is revealed here.
%\end{sol}

%\begin{sol}{prob2}
%\textbf{Problem Heading}\\
%(a) The solution of first part is revealed here.\\
%(b) The solution of second part is revealed here.
%\end{sol}

	%%%\input{geral/C.2.Referenc}

    % Pasta de bibliografia deve ser relativa à ``dist'':
    \bibliographystyle{\bibPasta/spphys}
	\bibliography{\bibPasta/bibliografia}
	\printindex

\end{document}
