\begin{partbacktext}
\part{Fundamentação Teórica}
%\noindent Use the template \emph{part.tex} together with the Springer document class SVMono (monograph-type books) or SVMult (edited books) to style your part title page and, if desired, a short introductory text (maximum one page) on its verso page in the Springer layout.

\end{partbacktext}


% Este tópico é essencial na obra, e recomendamos que seja elaborado com cuidado. Seu objetivo é o de, através de uma revisão criteriosa dos conceitos científicos fundamentais em que o material do livro se baseia, recordar os assuntos tratados aos leitores que já os conhecem e ao mesmo tempo, orientar conceitualmente os leitores que não possuam conhecimentos prévios dos assuntos pertinentes a cada uma das disciplinas envolvidas nos assuntos tratados, permitindo-lhes uma compreensão, ainda que básica, do tema tratado sem que necessitem acessar outras fontes de referência, tornando assim auto-contido o material apresentado no livro.

% Nele devem ser apresentados, com relação tanto às áreas básicas como às de aplicação, todos os elementos científicos que dão suporte à tecnologia que o livro pretende sugerir. Assim, sugere-se a inclusão de uma seção de definições, na qual se esclareça:

% o significado preciso de toda a terminologia empregada,

% o significado da simbologia utilizada nas notações formais utilizadas

% no caso de assuntos inter ou multidisciplinares, a exata relação entre os elementos das diversas disciplinas envolvidas

% Na composição dos assuntos reunidos sob a classe Fundamentos, subentendem-se elementos conceituais: teorias, bases matemáticas, notações formais, modelos de representação, teoremas, demonstrações, propriedades, relações de correspondência e de equivalência.

% Não menos importantes, devem ser apresentadas, quando couber, analogias com fenômenos naturais, ouabstratos, taiscomocomocomparaçõesgeométricas, algébricas, linguísticas, físicas, biológicas, sociológicas e quaisquer outras consideradas relevantes

% A leitura do tópico Fundamentos deve permitir que o leitor exercite e aprimore seu potencial de absorver os conteúdos dos capítulos tecnológicos subsequentes, ainda que não seja um especialista em qualquer das áreas de conhecimento tratadas na obra.
