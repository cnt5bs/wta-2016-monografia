%\motto{Use the template \emph{chapter.tex} to style the various elements of your chapter content.}
\section{Aplicações}
\label{cap:aplicacoes} % Always give a unique label
% use \chaptermark{}
% to alter or adjust the chapter heading in the running head

% Com base nos resultados apresentados na seção Tecnologia Empregada, tem-se agora um conjunto de problemas, associados às soluções consideradas mais indicadas. Para cada par problema-solução, esta seção deve apresentar exemplos ilustrativos, na forma de estudos de casos, nos quais se apresenta um detalhamento da resolução proposta, desenvolvido passo a passo, com explicação suficiente para que o experimento possa ser não apenas repetido como principalmente reproduzido em outras situações similares. O material contido nesta seção, portanto, deverá apresentar características que lhe permitam servir ao usuário como ferramenta para treinamento no uso dessa tecnologia, em problemas que surgem nas situações típicas do dia-a-dia.

\begin{quote}\textit{(A ideia desta seção é apresentar problemas e soluções que se baseiam no uso de MTS - UBA.)}\end{quote}

\subsection{Busca de informação em textos}

\begin{quote}\textit{(A ideia desta seção é apresentar o estudo de caso no qual o comportamento do animal Aplysia inspira no desenvolvimento de um modelo computacional habituação-sensibilização, posteriormente contextualizado em um padrão UBA-HS, utilizado para realizar busca de padrões em um problema de análise de texto.)}\end{quote}

\subsection{Narrador OCC-RDD}

\begin{quote}\textit{(A ideia desta seção é mostrar como os padrões UBA estão sendo utilizados como base para um projeto de pesquisa que se propõe a desenvolver uma ferramenta de apoio à narração de histórias OCC-RDD. Este tipo de história relaciona-se com a concepção de ambiente interativos de aprendizagem presencial em cursos de nível superior, foco de diversos outros projetos realizados pelo GEMS-PUCSP.)}\end{quote}

\subsection{Gerador de Textos OCC-RDD}

\begin{quote}\textit{(A ideia desta seção é apresentar uma arquitetura que foi utilizada para a geração automática de roteiros de histórias OCC-RDD e apontar revisões que façam uso de UBAs.)}\end{quote}

