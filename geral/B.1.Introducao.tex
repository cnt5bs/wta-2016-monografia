%\motto{Use the template \emph{chapter.tex} to style the various elements of your chapter content.}

% Resumo
%Um dos objetivos a serem alcançados pela leitura deste texto refere-se ao problema de elaborar modelos de comportamentos mecânicos bioinspirados. Tal elaboração pode ser realizada com o emprego do método MTS. Além disso, também há a preocupação referente com a construção de software de grande porte com comportamentos adaptativos. No texto, encontra-se uma proposta básica para lidar com este ponto: unidades biomiméticas adaptativas.

%Assim como a tecnologia adaptativa, a criação de sistemas de software biomimético também vem se desenvolvimento ao longo da última década. A novidade desta obra remete a um processo de desenvolvimento de software que procura sistematizar a etapa de análise e, por meio de padrões de modelagem, apontar questões sobre a elaboração de uma arquitetura de software adaptativa.

%O texto organiza-se em duas partes. Na primeira parte, apresentam-se os fundamentos do desenvolvimento de software bioinspirado, a tecnologia Adaptativa e as principais áreas de utilização. Na segunda parte, trata-se do desenvolvimento de software com MTS e UBA.

%Dentre as contribuições encontradas nesta publicação citam-se: uma notação para especificar autômatos adaptativos, um método para análise de software bioinspirado (MTS) e um catálogo de padrões biomiméticos que ajudam na elaboração de uma arquitetura de software adaptativa.

% Palavras-chave:
%adaptatividade
%biomimética
%unidades biomiméticas adaptativas
%modelagem de software
%semiótica

\chapter{Introdução}
\label{cap:intro} % Always give a unique label
% use \chaptermark{}
% to alter or adjust the chapter heading in the running head

%O capítulo de introdução deve conter informação suficiente para motivar o leitor a se interessar pela obra. Sugere-se que o capítulo de introdução contemple:

%motivação da obra
A ideia central desta monografia surgiu a partir dos resultados de pesquisa dos autores a respeito de elaboração de modelos de software biologicamente inspirados no contexto de arquiteturas adaptativas.

%objetivos da obra
Um dos objetivos a serem alcançados pela leitura deste texto refere-se ao problema de elaborar modelos de comportamentos mecânicos bioinspirados. Tal elaboração pode ser realizada com o emprego do método MTS. Além disso, também há a preocupação referente à construção de software de grande porte com comportamentos adaptativos. No texto, encontra-se uma proposta básica para lidar com este ponto: unidades biomiméticas adaptativas.

%apresentação geral das áreas de interesse

%histórico e estado da arte
Assim como a tecnologia adaptativa, a criação de sistemas de software biomimético também vem se desenvolvimento ao longo da última década. A novidade desta obra remete a um processo de desenvolvimento de software que procura sistematizar a etapa de análise e, por meio de padrões de modelagem, apontar questões sobre a elaboração de uma arquitetura de software adaptativa.

%metodologia adotada para alcançar os objetivos
Para alcançar aqueles objetivos, apresenta-se o método MTS, utilizado durante a atividade de análise visando a elaboração de um modelo comportamental mecânico, inspirado em animais que se adaptam ao mundo natural. Em seguida, uma revisão da tecnologia adaptativa é conduzida na parte de fundamentação. Durante esta revisão, propõe-se uma notação executável para especificar os diferentes modelos semânticos que auxiliam na concepção de uma linguagem de programação adaptativa, apropriada para a implementação de padrões biomiméticos. O texto organiza-se em três partes. Na primeira parte, apresentam-se os fundamentos do desenvolvimento de software bioinspirado e elementos centrais da tecnologia Adaptativa. Na segunda, descreve-se o projeto de um simulador de máquinas adaptativas voltadas para suportar o desenvolvimento de software baseado em unidades biomiméticas adaptativas (UBAs). Na terceira parte, trata-se da modelagem de software com MTS e UBA e das principais áreas de utilização.

%contribuições esperadas da obra
Dentre as contribuições encontradas nesta publicação citam-se: uma notação para especificar autômatos adaptativos, um método para análise de software bioinspirado (MTS) e um catálogo de padrões biomiméticos que ajudam na elaboração de uma arquitetura de software adaptativa.
