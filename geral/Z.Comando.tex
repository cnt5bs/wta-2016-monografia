
\usepackage{polyglossia}
\setmainlanguage{portuges}

% choose options for [] as required from the list
% in the Reference Guide
%\usepackage{mathptmx}
%\usepackage{helvet}
%\usepackage{courier}
\usepackage{xltxtra,fontspec,xunicode}
%\setmainfont[Ligatures=TeX]{Vollkorn}
%\setromanfont[Ligatures=TeX]{Vollkorn}
%\setsansfont[Ligatures=TeX,Scale=MatchLowercase]{DejaVu Sans}
%\setmonofont[Scale=MatchLowercase]{DejaVu Sans Mono}

\usepackage{amsmath}
\usepackage{amssymb}
\newfontfamily\listingsfont[Scale=0.7]{DejaVu Sans Mono}
\newfontfamily\listingsfontinline[Scale=0.8]{DejaVu Sans Mono}
\usepackage{listings}
	\usepackage{color}
	\definecolor{sh_comment}{rgb}{0.12, 0.38, 0.18 } %adjusted, in Eclipse: {0.25, 0.42, 0.30 } = #3F6A4D
	\definecolor{sh_keyword}{rgb}{0.37, 0.08, 0.25}  % #5F1441
	\definecolor{sh_string}{rgb}{0.06, 0.10, 0.98} % #101AF9
	\def\lstsmallmath{\leavevmode\ifmmode \scriptstyle \else  \fi}
	\def\lstsmallmathend{\leavevmode\ifmmode  \else  \fi}
	\lstset {
	language=Java,
%	frame=shadowbox,
	rulesepcolor=\color{black},
	showspaces=false,showtabs=false,tabsize=2,
%	numberstyle=\tiny,numbers=left,
	basicstyle= \listingsfont,
	stringstyle=\color{sh_string},
	keywordstyle = \color{sh_keyword}\bfseries,
	commentstyle=\color{sh_comment}\itshape,
	captionpos=b,
	xleftmargin=0.7cm, xrightmargin=0.5cm,
	lineskip=-0.3em,
	escapebegin={\lstsmallmath}, escapeend={\lstsmallmathend}
	}
	% Applies only when you use it
	\usepackage{xcolor}
	\lstdefinestyle{antlr}{
	basicstyle=\small\ttfamily\color{magenta},%
	breaklines=true,%                                      allow line breaks
	moredelim=[s][\color{green!50!black}\ttfamily]{'}{'},% single quotes in green
	moredelim=*[s][\color{black}\ttfamily]{options}{\}},%  options in black (until trailing })
	commentstyle={\color{gray}\itshape},%                  gray italics for comments
	morecomment=[l]{//},%                                  define // comment
	emph={%
	 STRING%                                            literal strings listed here
	 },emphstyle={\color{blue}\ttfamily},%              and formatted in blue
	alsoletter={:,|,;},%
	morekeywords={:,|,;},%                                 define the special characters
	keywordstyle={\color{black}},%                         and format them in black
	}
\usepackage{type1cm}
\usepackage{makeidx}         % allows index generation
\usepackage{graphicx}        % standard LaTeX graphics tool
                             % when including figure files
\usepackage{multicol}        % used for the two-column index
\usepackage[bottom]{footmisc}% places footnotes at page bottom

% Bibliography
%\usepackage{natbib,har2nat}
\usepackage[numbers]{natbib}
\renewcommand{\refname}{Bibliografia}

\usepackage[pdfpagelabels=true, plainpages=false,
            colorlinks=true, allcolors=blue]{hyperref}

% see the list of further useful packages
% in the Reference Guide
\makeindex             % used for the subject index
                       % please use the style svind.ist with
                       % your makeindex program

%\usepackage{epigraph}

\newcommand{\fig}[4][0.50]{\begin{figure}[ht]
%\sidecaption
\includegraphics[scale=#1]{#2/#3}
\caption{#4}
\label{#3}       % Give a unique label
\end{figure}}

\newcommand{\figBot}[4][0.50]{\begin{figure}[b]
%\sidecaption
\includegraphics[scale=#1]{#2/#3}
\caption{#4}
\label{#3}       % Give a unique label
\end{figure}}

\newcommand{\figTop}[4][0.50]{\begin{figure}[t]
%\sidecaption
\includegraphics[scale=#1]{#2/#3}
\caption{#4}
\label{#3}       % Give a unique label
\end{figure}}

\newcommand{\figLatHere}[4][0.50]{\begin{figure}[ht]
\sidecaption
\includegraphics[scale=#1]{#2/#3}\caption{#4}\label{#3}
\end{figure}}

\newcommand{\figLatTop}[4][0.50]{\begin{figure}[t]
\sidecaption
\includegraphics[scale=#1]{#2/#3}\caption{#4}\label{#3}
\end{figure}}

\newcommand{\figLat}[4][0.50]{\begin{figure}[b]
\sidecaption
\includegraphics[scale=#1]{#2/#3}\caption{#4}\label{#3}
\end{figure}}

%%% Elementos adaptativos
\newcommand{\naoTerminal}[1]{\lstinline[style=antlr]{#1}}
\newcommand{\terminal}[1]{`\lstinline[style=antlr]{#1}'}
\newcommand{\uml}[1]{\lstinline{#1}}
\newcommand{\codigo}[1]{\lstinline{#1}}
\newcommand{\uba}[1]{\lstinline{#1}}
\newcommand{\instancia}[1]{\lstinline{#1}}
\newcommand{\funcao}[1]{\lstinline{#1}}
\newcommand{\estado}[1]{\lstinline{#1}}
\newcommand{\simbolo}[1]{\lstinline{#1}}
\newcommand{\regra}[1]{``\lstinline{#1}''}
\newcommand{\acao}[1]{``\lstinline{#1}''}
%%%%%%%%%%%%%%%%%%%%%%%%%%%%%%%%%%%%%%%%%%%%%%%%%%%%%%

\newcommand{\bibPasta}{../../projeto-git/bib}
